\chapter{GUI-DSL}\label{GUI-DSL}
\section{Motivation des Ansatzes}
Eine \acrshort{GUI} erm�glicht die Interaktion mit einem Programm mit Hilfe
unterschiedlicher \acrshort{GUI}-Komponenten (vgl. \cite[S.4]{Galitz2007}). Mithilfe dieser Komponenten werden
Informationen dargestellt, oder Eingaben vom Nutzer get�tigt. Um die
Zusammensetzung dieser Komponenten zu beschreiben gibt
unterschiedliche Ans�tze, wovon zwei im Folgenden vorgestellt werden.
\begin{itemize}
  \item []
Beim ersten Ansatz wird die \acrshort{GUI} auf der Grundlage von fachlichen
Modellen generiert, was dem MDSD nahe kommt. \simplevcite{3} Das bedeutet, dass
in der Beschreibung der \acrshort{GUI} keine \acrshort{GUI}-Komponenten verwendet werden, wie es bei traditionellen
\acrshort{GUI}-Frameworks (JavaFX, Swing) der Fall ist.
\item []
Beim zweiter Ansatz wid ein Modell der \acrshort{GUI} beschrieben, welches in
andere GUI-Beschreibungen �berf�hrt werden kann .
Somit wird die Trennung zwischen \acrshort{GUI} und fachlichen Aspekten in der Software
erhalten und das Dom�nenproblem beschr�nkt sich auf die \acrshort{GUI}.
\end{itemize}
Die Komplexit�t des zweiten Ansatzes ist weitaus kleiner als die des ersten, da
lediglich die Aspekte, welche die \acrshort{GUI} betreffen, auf MDSD umgestellt
werden m�ssen. Au�erdem soll den Entwicklern weiterhin die M�glichkeit gegeben
werden, die \glspl{GUI} selbst zu entwerfen, was bei dem ersten Ansatz streng
genommen nicht m�glich ist, da die \acrshort{GUI} vollst�ndig generiert wird.
Beim zweiten Ansatz hingegen werden weiterhin \acrshort{GUI}-Komponenten in der
\acrshort{GUI}-Beschreibung verwendet, was dem Entwicklern die Anpassung der
\glspl{GUI} erm�glicht.\\
Von daher wird der zweite Ansatz weiter verfolgt und eine \g konzipiert, welche
die nachfolgenden Anforderung erf�llen soll.


\section{Anforderungen an die GUI-DSL}
Die allgemeinen Anforderung an die \acrshort{GUI} wurden in Kapitel \ref{AllgAnforderungen}
erl�utert. Die folgenden Anforderungen beziehen sich auf die
Ausdrucksm�glichkeit der \g, wodurch eine \acrshort{GUI} werden soll.
\begin{itemize}
  \item[\textbf{AS1}] Beschreibung von \glspl{GUI} �ber die
  Zusammensetzung von \acrshort{GUI}-Komponenten\label{AS1}
  \item[\textbf{AS2}] Wiederverwendung und Erweiterung bzw.
  Ver�nderung beschriebener
  \glspl{GUI}\label{AS2}
  \item[\textbf{AS3}] Verwendung einer abstrakten Layoutbeschreibung\label{AS3}
  \item[\textbf{AS4}] Weniger Quell-Code zur Beschreibung von
  \glspl{GUI}\label{AS4}
  \item[\textbf{AS5}] Beschreibung von Interaktionen an
  \acrshort{GUI}-Komponenten\label{AS5}
  \item[\textbf{AS6}] Erweiterung um neue \acrshort{GUI}-Komponenten\label{AS6}
\end{itemize}

\noindent
Wie im vorherigen Abschnitt bereits erw�hnt, sollen die Entwickler in der Lage
sein eine \acrshort{GUI} relativ frei zu gestalten. Das bedeutet, dass die
einzelnen \acrshort{GUI}-Komponenten unterschiedlich kombinierbar sein sollen
(\ref{AS1}). Au�erdem sollten die beschriebenen \glspl{GUI} auch in anderen
\glspl{GUI} wiederverwendet werde k�nnen (\ref{AS2})), da viele \glspl{GUI} in
profil c/s �hnlich sind.\\
Bez�glich des Layouts muss erw�hnt werden, dass in der traditionellen \acrshort{GUI}-Entwicklung
\glspl{GUI} die Strukturierung der \acrshort{GUI}-Komponenten mit Hilfe von
Layout-Containern vorgenommen wird.
In der Vergangenheit hat sich gezeigt, dass die Strukturierung �ber ein
spezifisches Layout zu einer Orientierung an ein bestimmtes Framework f�hrt (Beispiel: MCF orientiert sich an Swing). Das ist
ein Problem, da bestimmte Layouts auf anderen Plattformen (Bspw. Web oder
Mobil) nicht dargestellt werden k�nnen, da entsprechende Layoutmanager nicht
vorhanden sind.
Von daher ist das Layout in der \g so zu beschreiben, dass es auf
allen Plattformen gleicherma�en gut dargestellt werden kann (\ref{AS3}).\\
F�r eine Steigerung der Effizenz der deg ist bei der Einf�hrung neuer
Techniken wichtig, dass weniger Quell-Code geschrieben werden muss, als
zuvor, aber die Qualit�t nicht leidet (\ref{AS4}).\\
Bez�glich der \acrshort{GUI}-Komponenten ist nicht nur die Beschreibung der
Dargstellung, sondern aus die Beschreibung der Interaktionen von Belang
(\ref{AS5}).\\
Au�erdem darf die Erweiterung um neue \acrshort{GUI}-Komponenten nicht au�er
Acht gelassen werden (\ref{AS6}), da anderen Falls die Gefahr besteht, dass die
GUI-DSL unbrauchbar wird.

