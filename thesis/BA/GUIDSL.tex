\chapter{GUI-DSL}\label{GUI-DSL}
\section{Beschreibung der Anforderung an die GUI}
Die allgemeinen Anfroderung an die GUI wurden in Kapitel \ref{AllgAnforderungen}
erl�utert. Die folgenden Anforderungen beziehen sich auf die Aspekte der GUI die
beschrieben werden m�ssen, da sie von den Entwicklern im Verlauf der
Zeit ge�ndert werden m�ssen, oder es nicht sinnvoll ist diese zu abstrahieren,
da keine Wiederverwendung stattfindet. Bei der Beschreibung der Anforderungen
muss darauf eingegangen werden, welche Elemente der GUI mit anderen Elementen
kommunizieren k�nnen und wie die GUI beschrieben wird.\\
Traditionell werden GUIs mit Hilfe von Layout-Containern strukturiert. In der
Vergangenheit hat sich gezeigt, dass die strukturierung �ber ein spezifisches
Layout zu einer Orientierung an ein bestimmtes Framework f�hrt. Das ist ein
Problem, da bestimmte Layout im Web oder auf mobilen Plattformen nicht
genauso dargestellt werden k�nnen, wie auf einer Desktop-Anwendung. Von daher
ist das Layout in der GUI-Beschreibung so zu beschreiben, dass
das Layout auf allem Plattformen gleicherma�en gut dargestellt werden kann.\\
Neben dem Layout sind auch die Haupte lemente einer GUI-Struktur zu beachten.
Damit ist bspw. das Fenster f�r den Desktop gemeint. Im Web l�sst sich eine
solche Fenster-Sicht kaum darstellen. Bei der definition muss demnach auf eine
abstrakte Beschreibung solcher Elemente zur�ckgegriffen werden.\\
\gloss{Triviale UI-Komponenten}, die f�r profil c/s relevant sind, m�ssen
beschrieben werden k�nnen.
Welche Attribute dieser Elemente beschrieben werden, wird in Kaptiel \ref{Entwicklung
einer DSL zur Beschreibung der GUI in profil c/s} genauer analysiert. Im
Gegensatz dazu steht, dass die GUI durch die Beschreibung fachlicher Modelle
automatisch generiert wird und somit keine Elemente einer GUI beschrieben
werden m�ssen (Motivation f�r MDSD \cite{3}). Jedoch soll den Entwicklern
weiterhin die M�glichkeit gegeben werden, die UIs selbst zu entwerfen. Grund
daf�r ist, dass es ein zu gro�er Aufwand w�re alle Module von profil c/s auf
MDSD umzustellen und die GUIs generieren zu lassen. Nur die Beschreibung der GUI
umzustellen scheint �berschaubar zu sein.\\
Weiterhin m�ssen auch komplexe Elemente einer GUI beschrieben werden k�nnen.
Unter komplexen GUI-Elementen in profil c/s versteht man vordefinierte
Zusammensetzungen von trivialen GUI-Elementen, die f�r einen bestimmten Input
einen entsprechenden Output auf der Benutzerschnittstelle erzeugen. Die
Verarbeitung des Input ist dabei bereits an die Dom�ne angepasst. An diesem
Punkt kann die vorherige Anforderung wiederum angefochten werden, da komplexe
GUI-Elemente auf einer h�heren Abstraktionsebene liegen als triviale
GUI-Elemente. Die Anzahl der komplexen GUI-Elemente f�r profil c/s reicht jedoch
nicht, um alle Benutzerschnittstellen f�r profil c/s zu beschreiben. Au�erdem
ist die Wiederverwendbarkeit einiger komplex GUI-Elemente (gem�� dem Fall, dass
eine Vielzahl solcher Elemente best�nde und sich die Benutzerschnittstellen
von profil c/s damit abbilden lie�en) meiner Meinung nach fragw�rdig.\\
Da bereits Ans�tze von DSLs zur Beschreibung von GUIs existieren muss gepr�ft
werden, ob sie die oben beschreibenen Anforderungen umsetzen k�nnen. Dazu werden
im folgenden Kapitel drei ausgew�hlte DSLs vorgestellt und im nachfolgenden
Kapitel hinsichtlich der Anforderungen bewertet.
\section{Vorstellung ausgew�hlter DSLs zur Beschreibung von GUIs}
\subsection{The Snow}
\subsection{glc-dsl}
\subsection{Sculptor}
\section{Bewertung}
\begin{table}[!htb]
\centering
\small
			\begin{tabular}[c]{|l|r|r|r|}
			\hline
				\textbf{Anforderung} &
				\textbf{The Snow} & \textbf{glc-dsl} &
				\textbf{Sculptor} \\
				\hline\hline
				\textbf{abstraktes Layout} & & &\\
				\hline
				\textbf{abstrakte Hauptelemente} & & &\\
				\hline
				\textbf{Beschreibung trivialer Elemente} & 	& & \\
				\hline
				\textbf{Beschreibung komplexer Elemente} & & &\\
				\hline
			\end{tabular}
		\caption{Bewertung ausgew�hlter DSLs zur Beschreibung von GUIs}
\label{tab:guidsl}
	\end{table}\\