\chapter{GUI-DSL}\label{GUI-DSL}
\section{Motivation des Ansatzes}
Eine \gui erm�glicht die Interaktion mit einem Programm durch
unterschiedliche \emph{GUI-Komponenten} (vgl. \cite[S.4]{Galitz2007}). Mit
Hilfe dieser Komponenten werden Informationen dargestellt oder Eingaben vom
Nutzer get�tigt. Um die Zusammensetzung der \gui �ber eine \emph{DSL}
zu beschreiben, gibt unterschiedliche Ans�tze, wovon zwei im Folgenden vorgestellt
werden:
\begin{enumerate}
  \item 
Dieser Ansatz zeichnet sich dadurch aus, dass die \gui auf der
Grundlage von fachlichen Modellen generiert wird \simplevcite{3}. Das bedeutet,
dass in der Beschreibung der \emph{GUI-DSL} keine \emph{GUI-Komponenten}
verwendet werden, wie es bei \emph{traditionellen GUI-Frameworks}
(\emph{JavaFX}, \emph{Swing}) der Fall ist.
\item 
Der zweite Ansatz beschreibt ein Modell der \gui, welches
in andere \emph{GUI-Modelle} �berf�hrt werden kann. Somit wird die Trennung
zwischen der \gui und den fachlichen Aspekten in der Software erhalten und das
Dom�nenproblem auf die \gui reduziert.
\end{enumerate}
Die Umsetzung des letztgenannten Ansatzes ist weitaus einfacher, da
lediglich die Aspekte, welche die \gui betreffen, auf \emph{MDSD}
umgestellt werden m�ssen. Zudem soll den Entwicklern weiterhin die M�glichkeit
gegeben werden die \guis selbst zu entwerfen, was durch die im ersten
Ansatz genannte Beschreibungsform nicht m�glich ist,
da die \guis vollst�ndig generiert werden.
Wogegen unter der Verwendung des zweiten Ansatzes weiterhin
\emph{GUI-Komponenten} verwendet werden, was den Entwicklern die Anpassung
der \guis erm�glicht. Von daher wird dieser Ansatz weiter verfolgt und eine \g
konzipiert, welche die nachfolgenden Anforderungen erf�llen soll.


\section{Anforderungen an die GUI-DSL}
Die allgemeinen Anforderungen an die \gui und an die Sprache zur
Beschreibung dieser, wurden in Kapitel \ref{AllgAnforderungen} erl�utert. Die
folgenden Festlegungen beziehen sich auf die Ausdrucksm�glichkeit der
\emph{GUI-DSL}, wodurch eine \gui beschrieben werden soll.
\begin{itemize}
  \item [\textbf{AS1}] Beschreibung von \guis �ber die
  Zusammensetzung von \emph{GUI-Kompo\-nen\-ten}\label{AS1}
  \item [\textbf{AS2}] Wiederverwendung und Erweiterung bzw.
  Ver�nderung beschriebener
  \guis\label{AS2}
  \item [\textbf{AS3}] Verwendung einer abstrakten Layoutbeschreibung\label{AS3}
  \item [\textbf{AS4}] Gebrauch von weniger Quellcode zur Beschreibung von
  \guis\label{AS4}
  \item [\textbf{AS5}] Beschreibung von Interaktionen mit den
  \emph{GUI-Komponenten}\label{AS5}
  \item [\textbf{AS6}] Erweiterung um neue \emph{GUI-Komponenten}\label{AS6}
\end{itemize}

\noindent
Wie im vorherigen Abschnitt bereits erw�hnt, sollen die Entwickler in der Lage
sein, eine \gui relativ frei zu gestalten. Daraus folgt, dass
die einzelnen \gks unterschiedlich kombinierbar sein
m�ssen (siehe Anforderung \emph{\hyperref[AS1]{AS1}}). Zudem sollten die
beschriebenen
\guis auch in anderen \emph{GUI-DSL-Skripten (GUI-Skripten)}
wiederverwendet werden k�nnen (siehe Anforderung \emph{\hyperref[AS2]{AS2}})),
da viele \guis in \pcs �hnlich aufgebaut sind.\\
In Bezug auf das Layout muss erw�hnt werden, dass in der traditionellen
\emph{GUI-Entwicklung} die Strukturierung der \gks mit
Hilfe von Layoutcontainern vorgenommen wird.
In der Vergangenheit hat sich gezeigt, dass die Strukturierung �ber ein
spezifisches Layout zu einer Orientierung an einem bestimmten Framework f�hrt
(Beispiel: \MCF orientiert sich an Swing). Dies ist unvorteilhaft, da bestimmte
Layouts auf anderen Plattformen (bspw. Web oder Mobil) nicht dargestellt werden
k�nnen, weil entsprechende Layoutmanager nicht vorhanden sind. Somit ist
das Layout in der \g so zu beschreiben, dass es auf allen Plattformen gleicherma�en gut dargestellt werden kann
(siehe Anforderung \emph{\hyperref[AS3]{AS3}}).\\
F�r eine Steigerung der Effizienz in der \deg ist bei der Einf�hrung neuer
Technologien au�erdem darauf zu achten, dass weniger Quellcode geschrieben
werden muss als zuvor. Die Qualit�t darf darunter jedoch nicht leiden
(siehe Anforderung \emph{\hyperref[AS4]{AS4}}).\\
Da eine \gui ohne Interaktionsm�glichkeiten
ihren Zweck nicht erf�llen kann, ist die Beschreibung der Interkationen
unabdingbar (siehe Anforderung \emph{\hyperref[AS5]{AS5}}).\\
Dar�ber hinaus darf die Erweiterung um neue \gks nicht
vernachl�ssigt werden (siehe Anforderung \emph{\hyperref[AS6]{AS6}}), da
anderenfalls die Gefahr besteht, dass die \emph{GUI-DSL} unbrauchbar wird.

