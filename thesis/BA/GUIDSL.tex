\chapter{GUI-DSL}\label{GUI-DSL}
\section{Beschreibung der Anforderung an die GUI}
Die allgemeinen Anfroderung an die GUI wurden in Kapitel \ref{AllgAnforderungen}
erl�utert. Die folgenden Anforderungen beziehen sich auf die Aspekte der GUIs
die beschrieben werden m�ssen.
Ein UI erm�glicht die Interaktion mit einem Programm mit Hilfe unterschiedlicher
UI-Komponenten (vgl. \cite[S.4]{}). Mithilfe dieser Komponenten werden
Informationen dargestellt, oder Eingaben vom Nutzer get�tigt. Um die
Zusammensetzung dieser Komponenten zu beschreiben gibt es zwei Ans�tze.\\
Beim ersten Ansatz wird die GUI durch fachliche Modelle beschrieben. (Motivation
f�r MDSD \cite{3}) Das bedeutet, dass in der Beschreibung der GUI keine
UI-Komponenten, wie aus anderen UI-Frameworks bekannt ist, auftauchen. In der
deg soll den Entwicklern weiterhin die M�glichkeit gegeben werden, die UIs
selbst zu entwerfen. Grund daf�r ist, dass es ein zu gro�er Aufwand w�re alle
Module von profil c/s auf MDSD umzustellen und die GUIs generieren zu lassen.
Von daher ist dieser Ansatz vorerst nicht umsetzbar.\\
Die Komplexit�t des zweiten Ansatzes schein weitaus geringer zu sein. Dabei
werden weiterhin UI-Komponenten in der GUI-Beschreibung verwendet. Die
Entwickler haben somit die M�glichkeit die UIs in einem gewissen Grad
anzupassen.\\
F�r die GUI-DSL wird bzgl. der Komponenten zwischen drei Kategorien
unterschieden.\\
Die erste Kategorie sind umfasst \emph{triviale UI-Komponenten}. Dabei handelt
es sich um UI-Komponenten, deren Funktionen in unterschiedlichen UI-Frameworks 
�hnlich sind und allgemein einsetzbar sind. Das bedeutet, dass sie nicht als
dom�nenspezifisch angesehen werden k�nnen. Beispiele hierf�r sind UI-Komponenten
wie der \emph{Button} oder das \emph{Label}. Welche Attribute dieser Elemente
beschrieben werden, wird in Kaptiel \ref{Entwicklung einer DSL zur Beschreibung
der GUI in profil c/s} genauer analysiert.\\
Die zweite Kategorie umfasst \emph{komplexe UI-Komponenten}. Diese zeichnen sich
dadurch aus, dass sie dom�nenspezifisch sind und speziell f�r profil c/s
entwickelt wurden. Ein Beispiel hierf�r ist die
\gloss{Multiselection-Komponente}.\\
Die dritte Kategorie umfasse \emph{Layout Komponenten}. Dabei handelt es sich
um strukturgebende Komponenten. In anderen UI-Frameworks sind die bspw.
\emph{Panel}, \emph{Div} oder \emph{Pane}. In der GUI-DSL m�ssen auch solche
Komponenten verf�gbar sein. Dabei ist besonders auf die Ausdruckskraft der
f�r die Beschreibung dieser Komponenten verwendeten Bezeichnungen zu
achten\footnote{Auf dem Desktop lassen sich Fenster darstellen. In einem
Web-Browser ist der Begriff \emph{Fenster} nicht gel�uft. Die Komponente,
die hier dem Fester auf dem Desktop meiner Meinung nach gleicht, ist das
Tab.}.\\
Bez�glich des Layouts ist eine weitere Anforderung zu nennen. Hierzu muss
erw�hnt werden, dass in der traditionellen UI-Entiwcklung GUIs mit Hilfe von
Layout-Containern strukturiert werden. In der Vergangenheit hat sich gezeigt,
dass die strukturierung �ber ein spezifisches Layout zu einer Orientierung an
ein bestimmtes Framework f�hrt (Beispiel: MCF orientiert sich an Swing). Das ist
ein Problem, da bestimmte Layout im Web oder auf mobilen Plattformen nicht
genauso dargestellt werden k�nnen, wie auf einer Desktop-Anwendung. Von daher
ist das Layout in der GUI-Beschreibung so zu beschreiben, dass das Layout auf
allem Plattformen gleicherma�en gut dargestellt werden kann.\\
Dar�ber hinaus ist es f�r die Effizens in deg der wichtig, dass mit dem neuen
Ansatz vom GUI-Entwickler weniger Code geschrieben werden muss als mit
dem alten Ansatz.\\
Weiterhin ist f�r ein effizientes Arbeiten auch die Bereitstellung
eines Editors f�r die DSL von gro�er Wichtigkeit. Dieser Editor soll
nach M�glichkeit auch Validierungen durchf�hren k�nnen und
Code-Completion anbieten. Eine Integration dieses Editors in die von der deg
verwendete Entwicklungsumgebung (Eclipse) w�re dazu w�nschenswert.\\
Bez�glich der Anforderungen an die Komponenten ist anschlie�end zu sagen, dass
bei den trivialen und die komplexen UI-Komponenten die M�glichkeit bestehen muss
Interaktionen festzulegen. Au�erdem muss die GUI-DSL um weitere Komponenten
erweiterbar sein.

\section{Vorstellung ausgew�hlter DSLs zur Beschreibung von GUIs}
\subsection{The Snow}
\subsection{glc-dsl}
\subsection{Sculptor}
\section{Bewertung}
\begin{table}[!htb]
\centering
\small
			\begin{tabular}[c]{|l|r|r|r|}
			\hline
				\textbf{Anforderung} &
				\textbf{The Snow} & \textbf{glc-dsl} &
				\textbf{Sculptor} \\
				\hline\hline
				\textbf{abstraktes Layout} & & &\\
				\hline
				\textbf{abstrakte Hauptelemente} & & &\\
				\hline
				\textbf{Beschreibung trivialer Elemente} & 	& & \\
				\hline
				\textbf{Beschreibung komplexer Elemente} & & &\\
				\hline
			\end{tabular}
		\caption{Bewertung ausgew�hlter DSLs zur Beschreibung von GUIs}
\label{tab:guidsl}
	\end{table}\\