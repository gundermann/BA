\chapter{GUI-DSL}\label{GUI-DSL}
\section{Motivation des Ansatzes}
Eine \acrshort{GUI} erm�glicht die Interaktion mit einem Programm mit Hilfe
unterschiedlicher \acrshort{GUI}-Komponenten (vgl. \cite[S.4]{Galitz2007}). Mithilfe dieser Komponenten werden
Informationen dargestellt, oder Eingaben vom Nutzer get�tigt. Um die
Zusammensetzung der \acrshort{GUI} �ber eine DSL zu beschreiben gibt
unterschiedliche Ans�tze, wovon zwei im Folgenden vorgestellt werden.
\begin{itemize}
  \item []
Der erste Ansatz zeichnet sich dadurch aus, dass die \acrshort{GUI} auf der
Grundlage von fachlichen Modellen generiert wird. \simplevcite{3} Das bedeutet, dass in der Beschreibung der GUI-DSL keine
\acrshort{GUI}-Komponenten verwendet werden, wie es bei traditionellen \acrshort{GUI}-Frameworks (JavaFX, Swing) der Fall ist.
\item []
Der zweite Ansatz beschreibt ein Modell der \acrshort{GUI}, welches
in andere GUI-Modelle �berf�hrt werden kann. Somit wird die Trennung zwischen
der \acrshort{GUI} und den fachlichen Aspekten in der Software erhalten und das
Dom�nenproblem auf die \acrshort{GUI} reduziert.
\end{itemize}
Die Komplexit�t des zweiten Ansatzes ist weitaus geringer, da
lediglich die Aspekte, welche die \acrshort{GUI} betreffen, auf MDSD umgestellt
werden m�ssen. Au�erdem soll den Entwicklern weiterhin die M�glichkeit gegeben
werden, die \glspl{GUI} selbst zu entwerfen, was bei dem ersten Ansatz streng
genommen nicht m�glich ist, da die \acrshort{GUI} vollst�ndig generiert wird.
Beim zweiten Ansatz hingegen werden weiterhin \acrshort{GUI}-Komponenten verwendet, was dem Entwicklern die Anpassung der
\glspl{GUI} erm�glicht.\\
Von daher wird der zweite Ansatz weiter verfolgt und eine \g konzipiert, welche
die nachfolgenden Anforderung erf�llen soll.


\section{Anforderungen an die GUI-DSL}
Die allgemeinen Anforderung an die \acrshort{GUI} und an die Sprache zur
Beschreibung dieser wurden in Kapitel \ref{AllgAnforderungen} erl�utert. Die
folgenden Festlegungen beziehen sich auf die Ausdrucksm�glichkeit der \g,
wodurch eine \acrshort{GUI} beschrieben werden soll.
\begin{itemize}
  \item[\textbf{AS1}] Beschreibung von \glspl{GUI} �ber die
  Zusammensetzung von \acrshort{GUI}-Komponenten\label{AS1}
  \item[\textbf{AS2}] Wiederverwendung und Erweiterung bzw.
  Ver�nderung beschriebener
  \glspl{GUI}\label{AS2}
  \item[\textbf{AS3}] Verwendung einer abstrakten Layoutbeschreibung\label{AS3}
  \item[\textbf{AS4}] Weniger Quellcode zur Beschreibung von
  \glspl{GUI}\label{AS4}
  \item[\textbf{AS5}] Beschreibung von Interaktionen an
  \acrshort{GUI}-Komponenten\label{AS5}
  \item[\textbf{AS6}] Erweiterung um neue \acrshort{GUI}-Komponenten\label{AS6}
\end{itemize}

\noindent
Wie im vorherigen Abschnitt bereits erw�hnt, sollen die Entwickler in der Lage
sein eine \acrshort{GUI} relativ frei zu gestalten. Daraus folgt, dass
die einzelnen \acrshort{GUI}-Komponenten unterschiedlich kombinierbar sein
m�ssen (siehe Anforderung \ref{AS1}). Au�erdem sollten die beschriebenen
\glspl{GUI} auch in anderen GUI-DSL-Skripten (GUI-Skripten) wiederverwendet
werden k�nnen (siehe Anforderung \ref{AS2})), da viele \glspl{GUI} in profil c/s
�hnlich aufgebaut sind.\\
In Bezug auf das Layout muss erw�hnt werden, dass in der traditionellen
\acrshort{GUI}-Entwicklung die Strukturierung der \acrshort{GUI}-Komponenten mit Hilfe von
Layout-Containern vorgenommen wird.
In der Vergangenheit hat sich gezeigt, dass die Strukturierung �ber ein
spezifisches Layout zu einer Orientierung an ein bestimmtes Framework f�hrt (Beispiel: MCF orientiert sich an Swing). Das ist
nicht vorteilhaft, da bestimmte Layouts auf anderen Plattformen (Bspw. Web oder
Mobil) nicht dargestellt werden k�nnen, da entsprechende Layout-Manager nicht
vorhanden sind. Von daher ist das Layout in der \g so zu beschreiben, dass es
auf allen Plattformen gleicherma�en gut dargestellt werden kann
(siehe Anforderung \ref{AS3}).\\
F�r eine Steigerung der Effizienz der deg ist bei der Einf�hrung neuer
Technologien au�erdem darauf zu achten, dass weniger Quellcode geschrieben
werden muss, als zuvor. Die Qualit�t darf darunter jedoch nicht leiden
(siehe Anforderung \ref{AS4}).\\
Da eine \acrshort{GUI} ohne Interaktionsm�glichkeiten
ihren Zweck nicht erf�llen kann, ist die Beschreibung dieser neben der Angabe
von Informationen �ber die Darstellen ebenso von Belang (siehe
Anforderung \ref{AS5}).\\
Au�erdem darf die Erweiterung um neue \acrshort{GUI}-Komponenten nicht
vernachl�ssigt werden (siehe Anforderung \ref{AS6}), da anderen Falls die Gefahr
besteht, dass die GUI-DSL unbrauchbar wird.

