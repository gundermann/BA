\chapter{Motivation}
In der heutigen Zeit werden Programme auf vielen unterschiedlichen
Gr�ten\footnote{Desktop, Smartphone, Tablet} von ausgef�hrt. Die
Benutzeroberfl�che neben der internen Umsetzung immer ein wichtiger Faktor, der
f�r den Erfolg einer Anwendung eine gro�e Rolle spielt. \cite{1} Damit
einher geht die \gloss{Usability} einer Anwendung.
Denn eine \emph{[\ldots] schlechte Useability f�hrt zu Verwirrung und Miss- bzw.
Unverst�ndis beim Kunden}\cite{usability}.
Dadurch geht letztendlich Umsatz verloren.\\
Wenn ein Programm auf unterschiedlichen Gr�ten ausgef�hrt wird, muss der
Entwickler bei der traditionllen Entwicklung\footnote{Siehe Glossar:
\gls{Traditionelle UI-Entwicklung}} mehrere \gloss{Graphical User Interfaces (GUI)}
bereitstellen. Folglich werden mehrere GUIs mit unterschiedlichen
Toolkits oder Frameworks entworfen. Diese Framworks haben einen starken 
imperativen Character, sind schwer zu erweitern und sie verhalten sich
unterschiedlich abh�ngig von der speziellen Implementierung.\cite{10} Der
Entwickler muss das GUI bei diesem Ansatz f�r jedes Framework explizit
beschreiben.\\
Ein anderer Ansatz zur Beschreibung von Benutzeroberfl�chen ist das Model-Driven
Development. Damit sollen UIs anhand der implementierten Funktionen automatisch erzeugt werden k�nnen. Allerdings wird die Darstellung dieser generierten UIs
von der Darstellung von traditionell implementierter Benutzerschnittstellen
�bertroffen.\cite{20}.\\
Eine �berlegung, die sich daraus ergibt, ist, ob man diese beiden Ans�tze zur
Implementierung von UIs (traditionell und Model-Driven) verbinden kann. Somit
kann die genaue Beschreibung der Darstellung mit einer h�heren Abstraktion
verbunden werden.\\
In dieser Arbeit wird versucht diese Idee umzusetzen. Bei der Umsetzung wird 
sich auf die UIs der Anwendung
\emph{profil c/s}. Profil c/s ist \gloss{INVEKOS}-Programm welches von der deg
als Client-Server-Anwendung entwickelt wird.
In dieser Arbeit wird versucht diese Idee an einem ausgew�hlten
Beispiel umzusetzen.

