\chapter{Motivation}
In der heutigen Zeit werden Programme auf vielen unterschiedlichen
Gr�ten (bspw. Desktop, Smartphone, Tablet) ausgef�hrt. Die
Benutzeroberfl�che ist neben der internen Umsetzung immer ein wichtiger Faktor,
der f�r den Erfolg einer Anwendung eine gro�e Rolle spielt. \cite{1} Ein
wichtige Eigenschaft der Benutzerschnittstelle bzgl. dessen ist die \gloss{Usability}.
Grund daf�r ist, dass eine \emph{[\ldots] schlechte Useability f�hrt zu
Verwirrung und Miss- bzw.
Unverst�ndis beim Kunden}\cite{usability}.
Dadurch geht letztendlich Umsatz verloren.\\
Wenn ein Programm auf unterschiedlichen Gr�ten ausgef�hrt wird, muss der
Entwickler bei der traditionllen Entwicklung\footnote{Siehe Glossar:
\gls{Traditionelle UI-Entwicklung}} mehrere \gloss{Graphical User Interfaces (GUI)}
manuell implementieren. Folglich werden mehrere GUIs mit unterschiedlichen
Toolkits oder Frameworks entworfen. Diese Framworks haben einen starken 
imperativen Character, sind schwer zu erweitern und sie verhalten sich
unterschiedlich abh�ngig von der speziellen Implementierung.\cite{10} Daraus
folgt, dass Entwickler bei dem traditionellen Ansatz das GUI bei diesem Ansatz
f�r jedes Framework explizit beschreiben muss.\\
Ein anderer Ansatz zur Beschreibung von Benutzeroberfl�chen ist das Model-Driven
Development (siehe \ref{mdsd}). Damit sollen bspw. UIs anhand der
modelierten Funktionalit�ten automatisch erzeugt werden k�nnen. \cite{3}
Laut Myers et. al. wurde die Darstellung dieser generierten UIs in der
Vergangenheit von der Darstellung von traditionell implementierter
Benutzerschnittstellen �bertroffen.\cite{20} Grund daf�r ist, dass die Abstraktionsebene, auf der die
GUIs beschrieben wurden, oft nicht an die Gestaltung der GUI, sondern an
fachlichen Konzepte der Dom�ne angepasst wurde.\\
Daraus ergibt sich die �berlegung, ob diese
beiden Ans�tze zur Implementierung von UIs (traditionell und Model-Driven)
verbunden werden k�nnen.
Somit k�nnte die genaue Beschreibung der Darstellung mit einer h�heren
Abstraktion verbunden werden. Dieser Versuche wurde beispielsweise von
Bacikov{\'a} im Jahr 2013 (siehe \cite{11}) unternommen. Dort wurde nachgewiesen,
dass ein GUI eine Sprache definiert.\\
In dieser Arbeit wird versucht diese Idee in einem anderen
Fachbereich umzusetzen, um die Unterst�tzung eines GUIs auf unterschiedlichen
PLatformen zu gew�hrleisten.
Bei der Umsetzung wird sich auf die UIs der Anwendung
\emph{profil c/s}. Profil c/s ist eine JEE-Anwendung die
\gloss{INVEKOS} umsetzt und von der der deg als entwickelt wird.

