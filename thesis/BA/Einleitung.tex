\chapter{Einleitung}
\section{Motivation}
In der heutigen Zeit werden Webseiten und Programme auf vielen unterschiedlichen
Gr�ten von unterschiedlichen Nutzern ausgef�hrt. So gut wie die interne
Umsetzung einer Anwendung auch sein mag, ist die \gls{Usability} einer Anwendung
immer ein sehr wichtiger Aspekt. Denn eine \emph{schlechte Useability
f�hrt zu Verwirrung und Miss- bzw. Unverst�ndis beim Kunden}\cite{usability}.
Dadurch geht letztendlich Umsatz verloren.\\
Bei der Entwicklung von Benutzeroberfl�chen muss aber wie Eingangs erw�hnt
darauf geachtet werden, dass diese auch auf unterschiedlichen
Ger�ten\footnote{Desktop, Smartphone, Tablet}. Das hat f�r den Entwickler in der
Regel zur Folge, dass dieser mehrere \gls{Graphical User Interfaces (GUI)}
bereitstellen muss. Somit werden mehrere GUIs entworfen, die dieselben Aspekte
abbilden sollen. Von stellt sich die Frage, ob es nicht m�glich ist, ein GUI in
einer speziellen Sprache so abstrakt zu beschreiben, dass einmal beschrieben f�r
alle unterschiedlichen Ger�te exportiert werden kann.\\
In dieser Arbeit wird versucht diese Idee
an einem Beispiel umzusetzen. Das vorgesehene Beispiel wird in Kaptiel
\ref{Analyse} erkl�rt. Vorerst wird jedoch die Vorgehensweise zur Umsetzung der
Idee in dieser Arbeit erl�utert.
\cite{voelter}
\section{Vorgehen}
Die \emph{data experts GmbH (deg)} hat f�r die Umsetzung von GUIs auf
unterschiedlichen Plattformen eine L�sung entwickelt. In Kaptiel \ref{Analyse}
dieser Arbeit wird zun�cht dieser Ist-Zustand in der deg analysiert und die
Probleme aufgezeigt.\\
 Im nachfolgenden Kapitel (\ref{Dom�nenspezifische
Sprachen}) wird auf Dom�nenspezifische Sprachen (\gls{DSL}) eingegangen. Dies
ist wichtig, da die Umsetzung einer neuen Sprache zur abstrakten Beschreibung
eines GUI auf dem Konzept der DSLs beruht.\\
 F�r die Entwicklung von GUI gibt es
bereits einige DSLs. Aus diesem Grund wird in Kapitel \ref{Notwendigkeit einer
neuen DSL} darauf eingegangen, warum die deg nicht mit den bestehenden DSLs zur
GUI-Beschreibung arbeiten sollte.\\ 
Die Features, die eine DSL f�r die Beschreibung von GUIs f�r die deg
ben�tigt, werden im darauf folgenden Kapitel (\ref{Grobkonzept der
Sprache und eines Generators}) erl�utert. Da mit einer DSL alleine
noch keine Umsetzung eines Programms erm�glicht, werden in diesem Kapitel
auch die notwenigen Features von Generatoren betrachtet, die zur Generierung
von Quellcode ben�tigt werden.\\
Darauf aufbauend wird zuerst im Kapitel 
\ref{Entwerfen einer DSL zur Beschreibung der GUI in profil c/s} auf die
konkrete Umsetzung der DSL eingegangen. Die GUIs der deg sollten damit
ausreichend beschrieben werden k�nnen. Eine Pr�fung dessen wird sp�ter statt
finden.\\
Nachfolgen (Kapitel \ref{Entwicklung des Generators f�r das Generieren von Klassen f�r das Multichannel-Framework})
 wird ein Generator f�r diese DSl entwickelt. Somit kann aus der
 Beschreibung der GUI Quellcode generiert werden, der in der deg eingesetzt
 und getestet werden kann.\\
 Zum Abschluss (Kapitel \ref{Zusammenfassung und Ausblick}) werden die Ergebnisse zusammengafasst und ein Ausblick gegeben.
