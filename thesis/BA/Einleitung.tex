\chapter{Motivation}
In der heutigen Zeit werden Programme auf vielen unterschiedlichen
Ger�ten (bspw. Desktop, Smartphone, Tablet) ausgef�hrt. Die \gloss{Useability}
ist ein wichtiger Faktor bei der Entwicklung von Anwendungen. 
Denn \simpledcite{schlechte Usability f�hrt zu Verwirrung und Miss- bzw.
Unverst�ndnis}{usability} beim Kunden, wodurch letztendlich Umsatz verloren
geht. Die \gloss{Useability} wird haupts�chlich vom \glossLink{GUI}{Graphical User
Inferface (GUI)} bestimmt. Folglich ist die Benutzeroberfl�che neben der
internen Umsetzung ein wichtiger Faktor f�r den Erfolg einer Anwendung
\simplevcite{1}.
\\
Wenn ein Programm auf unterschiedlichen Ger�ten ausgef�hrt wird, muss der
Entwickler bei der \glossLink{Traditionlle GUI-Entwicklung}{traditionellen
GUI-Entwicklung} mehrere \emph{\glspl{GUI}} manuell implementieren.
Folglich werden mehrere \emph{\glspl{GUI}} mit unterschiedlichen
Toolkits oder Frameworks entworfen. Diese Framworks haben einen starken
imperativen Charakter, sind schwer zu erweitern und verhalten sich je nach
Plattform unterschiedlich. \simplevcite{10} Daraus folgt, dass Entwickler bei
dem \emph{traditionellen} Ansatz das \acrshort{GUI} f�r jedes Framework explizit
beschreiben m�ssen.\\
Ein anderer Ansatz zur Beschreibung von Benutzeroberfl�chen ist das
\emph{Model-Driven Development} (siehe Kapitel \ref{mdsd}). Damit sollen bspw.
\emph{\glspl{GUI}} anhand der modellierten Funktionalit�ten automatisch erzeugt
werden \simplevcite{3}. Laut Myers et. al. wurden die Darstellungen dieser
generierten \emph{\glspl{GUI}} in der Vergangenheit von den Darstellungen
\emph{traditionell} implementierter Benutzerschnittstellen �bertroffen
\simplevcite{20}. Grund daf�r ist, dass die Abstraktionsebene, auf der die
Beschreibung der \emph{\glspl{GUI}} beim \emph{Model-Driven Development}
stattfindet, oft nicht an die Gestaltung der \emph{\acrshort{GUI}}, sondern an
fachliche Konzepte der \gloss{Dom�ne} angepasst wird.\\
Daraus ergibt sich die �berlegung, ob diese
beiden Ans�tze zur Implementierung von \emph{\glspl{GUI}} (\emph{traditionell}
und \emph{Model-Driven}) verbunden werden k�nnen (\emph{kombinierter Ansatz}).
Somit k�nnte die genaue Beschreibung der Darstellung mit einer h�heren
Abstraktion verbunden werden. Dieser Versuch wurde bspw. von
Bacikov{\'a} im Jahr 2013 unternommen \simplevcite{11}. Dort wurde
nachgewiesen, dass eine \emph{\acrshort{GUI}} eine Sprache definiert.\\
In dieser Arbeit wird versucht den kombinierten Ansatz in einem speziellen
Fachbereich umzusetzen, um die Unterst�tzung einer \emph{\acrshort{GUI}} auf
unterschiedlichen Plattformen zu gew�hrleisten.
Bei der Umsetzung wird sich auf die \emph{\glspl{GUI}} der Anwendung
\emph{profil c/s} bezogen. Profil c/s ist eine \gloss{JEE}-Anwendung die
\gloss{InVeKoS} umsetzt und von der \emph{data experts gmbh (deg)} entwickelt
wird.

