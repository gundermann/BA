\chapter{Motivation}
In der heutigen Zeit werden Programme auf vielen unterschiedlichen
Ger�ten (bspw. Desktop, Smartphone, Tablet) ausgef�hrt. Die \gloss{Usability}
ist ein wichtiger Faktor bei der Entwicklung einer Anwendung. 
Denn eine \emph{[\ldots] schlechte Useability f�hrt zu
Verwirrung und Miss- bzw.
Unverst�ndis beim Kunden.} \cite{usability} Dadurch geht letztendlich Umsatz
verloren. Die Usability wird haupts�chlich von der Benutzerschittstelle
bestimmt. Folglich ist die Benutzeroberfl�che neben der internen Umsetzung
immer ein wichtiger Faktor, der f�r den Erfolg einer Anwendung eine gro�e Rolle
spielt. \simplevcite{1}
\\
Wenn ein Programm auf unterschiedlichen Gr�ten ausgef�hrt wird, muss der
Entwickler bei der UI-Entwicklung traditionllen UI-Entwicklung mehrere
Graphical User Interfaces (GUIs) manuell implementieren. Folglich werden mehrere
GUIs mit unterschiedlichen Toolkits oder Frameworks entworfen. Diese Framworks haben einen starken 
imperativen Charackter, sind schwer zu erweitern und sie verhalten sich je nach
Platform unterschiedlich. \simplevcite{10}
Daraus folgt, dass Entwickler bei dem traditionellen Ansatz das GUI f�r jedes
Framework explizit beschreiben muss.\\
Ein anderer Ansatz zur Beschreibung von Benutzeroberfl�chen ist das Model-Driven
Development (siehe Kapitel \ref{mdsd}). Damit sollen bspw. UIs anhand der
modelierten Funktionalit�ten automatisch erzeugt werden. \simplevcite{3}
Laut Myers et. al. wurde die Darstellung dieser generierten User Interfaces
(UIs) in der Vergangenheit von der Darstellung traditionell implementierter
Benutzerschnittstellen �bertroffen.\simplevcite{20} Grund daf�r ist, dass die
Abstraktionsebene, auf der die GUIs beschrieben wurden, oft nicht an die Gestaltung der GUI, sondern an
fachliche Konzepte der Dom�ne angepasst wurde.\\
Daraus ergibt sich die �berlegung, ob diese
beiden Ans�tze zur Implementierung von UIs (traditionell und Model-Driven)
verbunden werden k�nnen (kombinierter Ansatz).
Somit k�nnte die genaue Beschreibung der Darstellung mit einer h�heren
Abstraktion verbunden werden. Dieser Versuch wurde bspw. von
Bacikov{\'a} im Jahr 2013 unternommen. \simplevcite{11} Dort wurde
nachgewiesen, dass ein GUI eine Sprache definiert.\\
In dieser Arbeit wird versucht den kombinierten Ansatz in einem anderen
Fachbereich umzusetzen, um die Unterst�tzung eines GUIs auf unterschiedlichen
Plattformen zu gew�hrleisten.
Bei der Umsetzung wird sich auf die UIs der Anwendung
\emph{profil c/s}. Profil c/s ist eine \gloss{JEE}-Anwendung die
\gloss{InVeKoS} umsetzt und von der deg entwickelt wird.

