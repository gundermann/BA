\chapter{Evaluation des Frameworks zur Entwicklung der DSL}\label{Evaluation des Frameworks zur Entwicklung der DSL}
\section{Vorstellung ausgew�hlter Frameworks}
Zur Umsetzung der DSL und der Generatoren wird ein Framework ben�tigt, welches
daf�r notwendige Funktionalit�ten bereit stellt. Hierzu werden die Frameworks
\emph{PetitParser}, \emph{Xtext} und \emph{MPS} kurz vorgestellt und im
Anschluss verglichen.
\subsection{PetitParser}
Dieses Frameworks arbeitet mit Parser-Kombinatoren. Somit ist es mit
PatitParser einfach Grammatiken zusammenzustellen, zu transformieren oder zu
erweitern, sowie Teile dieser dynamisch wiederzuverwenden. Alles
geschieht auf der Basis von Pharo Smalltalk, womit das Framework urspr�nglich
implementiert wurde. Es existieren jedoch auch Versionen des Frameworks f�r
Java\footnote{https://github.com/petitparser/java-petitparser},
Dart\footnote{https://github.com/petitparser/dart-petitparser} und
PHP\footnote{https://github.com/mindplay-dk/petitparserphp}. Einfache Parser
bestehen aus Sequenzen von Funktionen, welche die Produktionsregeln
(Produktionen) der Grammatik abbilden. Komplexere Parser werden durch die
Kombination anderer Parser implementiert. (vgl. \cite{Renggli}) Die Kombination kann
in einer einzelne Methode implementiert werden, was dazu f�hrt, dass man den
Parser in einer Skript-Form erh�lt. Alternativ k�nnen die zu kombinierenden
Parser auch in Methoden von Unterklassen des PetitParsers implementiert werden.
(\cite[S.6]{brauer2010dsl}) Das f�rdert die Lesbarkeit, �berschichtlichkeit und schlie�lich
die Wartbarkeit des Codes.\\
Tool Support ist f�r dieses Frame gew�hrleistet. Mithilfe dessen k�nnen
Produktions editiert und grafisch abgebeildet werden. Aus Zufallsbeispiele
f�r ausgew�hlte Produktionen werden generiert um so Fehler in der Grammatik
aufzudecken. Dar�berhinaus wird die Effiziens einer Grammatik durch die
Darstellung direkter, ineffizienter Zyklen in der Grammatik verbessert.
(vgl. \cite{Renggli})
\subsection{Xtext}
Bei \emph{Xtext} handelt es sich um eine Open-Source-L�sung f�r einen
\emph{ANTLR}-basierten Parser- und Editorgenerator mit der externe, textuelle
DSLs entwickelt werden k�nnen.
Die Grammatiken f�r den Parser-Generator werden in der EBNF definiert. Durch die
Intergration in Eclipse kann der Eclipse-Editor f�r s�mtliche Artefakte der
Infrastruktur verwendet werden. Aus der Grammatik wir der LL(k)-Parser,
sowie ein Model, mit dessen Hilfe der Generator implementiert
werden kann, generiert. Die Klassen f�r Validierungs-Reglen und den Generator
werden ebenfalls vom Tool erzeugt. Diese m�ssen im Anschluss daran vom Nutzer
entsprechend erweitert werden. Zur Editierung der entsprechenden Dateien wird
eine eigene Syntax verwendet, die meiner Meinung jedoch stark an die Java-Syntax erinnert.\\
Wenn der Parser generiert wurde, ist Xtext in der Lage einen in Eclipse
integrierten Editor zu erzeugen. \cite[S.1]{carpediem} Dieser Editor ist in
der Lage die Validierungs-Regeln auf die DSL-Scripte anzuwenden. Dar�ber
hinaus wird auch Code-Completion vom Editor angeboten.
\subsection{MPS}
\section{Vergleich und Bewertung der vorgestellten
Frameworks}\label{vergleichFramework}
Know-How
Erlernbarkeit
Machbarkeit der Intergration
Verwendung eines Editors f�r DSL-Scripte
Erweiterbarkeit der Grammatik
Erweiterbarkeit der Validierungen
