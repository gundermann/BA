\chapter{Hinf�hrung zum Thema}\label{Analyse}
Da die deg bereits eine L�sung entwickelt hat, mit der es m�glich ist, ein
einmal entwickeltes GUI auf mehreren Plattformen darzustellen, muss voerst
analysiert werden, warum diese L�sung nicht mehr zukunftstauglich ist.\\
Bei s�mtlichen GUIs wird sich auf GUIs des Programms \emph{profil c/s} bezogen.
Profil c/s ist \gls{INVEKOS}-Programm welches von der deg als
Client-Server-Anwendung entwickelt wird.

\section{Allgemeine Anforderungen an Benutzeroberfl�chen von profil c/s}
Die Anforderung, welche zu der Entwicklung der eingangs kurz erw�hnten L�sung
gef�hrt hat, ist dass der Client von profil c/s sowohl in
Web-Browsern (Web-Client) als auch standalone auf einem
PC (Standalone-Client) ausgef�hrt werden soll. Eine weitere
Anforderung ist es, dass beide Clients (Web-Client und
Standalone-Client) eine �hnlichen Aufbau haben.\\
In Abbildung \ref{Abb_Web-Client} und Abbildung
\ref{Abb_Standalone-Client} ist das GUI eines \gls{Zuwendungsblatt}es eines
\gls{F�rderantrag}s zu sehen.
\myHugeFigure{web.jpg}{Web-Client: Zuewndungsblatt\cite{DegDKElerIAntragsmappe}}{Web-Client}
\myBigFigure{standalone.jpg}{Standalone-Client:
Zuewndungsblatt\cite{DegDKElerIAntragsmappe}}{Standalone-Client} F�r den Aufbau sind nur die Tabelle und
die darunter stehenden Buttons, sowies das Bemerkungsfeld (im Web-Client auf der rechten Seite und im
Standalone-Client in der Mitte) von Bedeutung.
Insofern wurde die Anforderung bzgl.
des gleichen Aufbaus umgesetzt, auch wenn es beim Vergleich der beiden GUIs im
ersten Moment nicht so aussieht.
\newline
\newline
Im folgenden Kapitel wird erkl�rt, wie die
deg diese Anforderungen umgesetzt hat.
\section{Umsetzung der
Benutzerschnittstellen f�r mehreren Plattformen in der deg (Ist-Zustand)}
F�r die Umsetzung dieser Anforderungen w�re es m�glich gewesen f�r den
Web-Client und dem Standalone-Client separate GUIs zu entwickeln. Die deg hat
jedoch eine L�sung erarbeitet mit der es m�glich ist, ein einmal
beschriebenes GUI auf mehrere Plattformen zu protieren. Das reduziert den
Aufwand zur Entwicklung neuer GUIs. Die L�sung der deg ist das
\emph{Multichannel-Framework}.\\
Innerhalb dieses Frameworks werden die GUIs mittels so genannter
\emph{Pr�sentationsformen} beschreiben. Die Architektur des
Multichannel-Frameworks ist Abbildung \ref{Abb_MC-Framework} zu entnehmen. 
\myBigFigure{mcgrob.png}{Architektur des
Multichannel-Frameworks\cite{jwammc}}{MC-Framework}
Daraus wird deutlich, dass aus Pr�sentationsformen mithilfe der
\emph{Component-Factories} GUIs erzeugt werden, die auf unterschiedlichen
Frameworks basieren\footnote{Hier: Swing, ULC und WingS. Wobei ULC bei der deg
nicht mehr im Einsatz ist.}. Somit ist die deg in der Lage ihre GUIs f�r das
\emph{\gls{Swing}}-Framework und f�r das \emph{\gls{WingS}}-Framework mit nur
einer GUI-Beschreibung zu erzeugen.

\section{Probleme des Multichannel-Frameworks}
Beim Einsatz des Multichannel-Frameworks treten jedoch zwei gro�e Probleme auf.
Das erste Problem bezieht sich auf die integrierten Framworks (\gls{Swing} und
\gls{WingS}). Beide Frameworks sind verwaltet und werden nicht mehr gewartet.
Um auch in der Zukunft den Anforderungen der Kunden nachkommen zu k�nnen m�ssten
beide Frameworks von der deg weiterentwickelt werden, da diese Aufgabe aus
unterschiedlichen Gr�nden von den Entwicklern der Frameworks nicht mehr
wahrgenommen wird. Eine andere M�glichkeit w�re es, wenn die deg andere und
modernere Frameworks einsetzt um den n�tigen Support der Framework-Entwickler
nutzen zu k�nnen.\\
Das Multichannel-Framework ist in der Theorie so konzipiert, dass es leicht sein
sollte neue Frameworks zu integrieren (siehe Abbildung \ref{Abb_MC-Framework}.
In der Praxis hat sich jedoch gezeigt, dass es nicht so einfach ist. Das
Problem, welches bei der Integration neuer Frameworks aufkommt, ist, dass sich
das Multichannel-Framework sehr stark an Swing orientiert und die GUIs vor allem
vom \gls{GridBagLayout} stark beeinflusst sind. Ein solches Layout steht nicht
in allen Frameworks zur Verf�gung. Da die Beschreibung �ber ein solches Layout
statt findet und der Aufbau der GUIs unterschiedlicher Frameworks per
Anforderung gleich sein soll, ist es die Verf�gbarkeit eines
\gls{GridBagLayout}s somit eine Voraussetzung f�r die Intergration in das
Multichannel-Framework.

\section{Kurzvorstellung der L�sung via DSL}
Aufgrund der nur schwer machbaren Integration neuer Frameworks in das
bestehende Multichannel-Framework und der Tatsache, dass die derzeit genutzten
Frameworks (\gls{Swing} und \gls{WingS}) veraltet sind, wird ein neuer Ansatz f�r die
Umsetzung von GUIs auf unterschiedlichen Plattformen gesucht.
\myBigFigure{neuerAnsatz.png}{DSL-Ansatz f�r gleich GUIs auf
unterschiedlichen Plattformen}{neuerAnsatz}\\
Der neue Ansatz basiert auf der folgenden Idee. Die GUIs sollen weiterhin nur
einmal beschrieben werden sollen. Diese Beschreibung soll �ber eine DSL
erfolgen und sich nicht an bestehende Frameworks orientieren. Grund daf�r ist,
dass ansonsten die Gefahr besteht, dass langfristig betrachtet mit diesem Ansatz
das gleiche Problem auftirtt wie beim Multichannel-Framework. Aus der
Beschreibung der GUIs wird ein Generator speziellen Quellcode erzeugen, der sich
auf entsprechenden Plattformen ausf�hren l�sst. F�r jedes eingesetzte Framework
muss somit ein eigener Generator entwickelt werden. Abbildung \ref{Abb_neuerAnsatz}
bildet die aus dieser Idee resultierende Architektur ab\footnote{Hier: Vaadin als
Web-Framework, JavaFX als Framework f�r den Standalone-Client und
Android als Repr�sentant f�r einen m�glichen Mobile-Client}.

