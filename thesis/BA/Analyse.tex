\chapter{Analyse des Ist-Zustands}\label{Analyse}
Da die deg bereits eine L�sung entwickelt hat, mit der es m�glich ist, ein
einmal entwickeltes GUI auf mehreren Plattformen darzustellen, muss voerst
analysiert werden, warum diese L�sung nicht mehr zukunftstauglich ist.\\
Bei s�mtlichen GUIs wird sich auf GUIs des Programms \emph{profil c/s} bezogen.
Profil c/s ist \gls{INVEKOS}-Programm welches von der deg als
Client-Server-Anwendung entwickelt wird.

\section{Allgemeine Anforderungen an Benutzeroberfl�chen von profil c/s}
Die Anforderung, welche zu der Entwicklung der eingangs kurz erw�hnten L�sung
gef�hrt hat, ist dass der Client von profil c/s sowohl in
Web-Browsern (Web-Client) als auch standalone auf einem
PC (Standalone-Client) ausgef�hrt werden soll. Eine weitere
Anforderung ist es, dass beide Clients (Web-Client und
Standalone-Client) eine �hnlichen Aufbau haben.\\
In Abbildung \ref{Abb_Web-Client} und Abbildung
\ref{Abb_Standalone-Client} ist das GUI eines \gls{Zuwendungsblatt}es eines
\gls{F�rderantrag}s zu sehen.
\myHugeFigure{web.jpg}{Web-Client: Zuewndungsblatt}{Web-Client}
\myBigFigure{standalone.jpg}{Standalone-Client: Zuewndungsblatt}{Standalone-Client}
F�r den Aufbau sind nur die Tabelle und die darunter stehenden Buttons, sowies
das Bemerkungsfeld (im Web-Client auf der rechten Seite und im
Standalone-Client in der Mitte) von Bedeutung.
Insofern wurde die Anforderung bzgl.
des gleichen Aufbaus umgesetzt, auch wenn es beim Vergleich der beiden GUIs im
ersten Moment nicht so aussieht.

\newpage
Im folgenden Kapitel wird erkl�rt, wie die
deg diese Anforderungen umgesetzt hat.
\section{Umsetzung der
Benutzerschnittstellen f�r mehreren Plattformen in der deg}
F�r die Umsetzung dieser Anforderungen w�re es m�glich gewesen f�r den
Web-Client und dem Standalone-Client separate GUIs zu entwickeln. 

\section{Probleme des Multichannel-Frameworks}