\chapter{Hinf�hrung zum Thema}\label{Analyse}


\section{Allgemeine Anforderungen an Benutzeroberfl�chen von profil c/s}
Die wichtigen (prim�ren) Anforderungen f�r diese Arbeit beziehen sich auf den
Client von profil c/s. Dieser soll sowohl in Web-Browsern (Web-Client) als auch standalone
auf einem PC (Standalone-Client) ausgef�hrt werden k�nnen. Weiterhin war es
f�r die deg wichtig, dass der Aufbau der UIs im Web- und Standalone-Client
�hnlich ist.\\
In Abbildung \ref{Abb_Web-Client} und Abbildung \ref{Abb_Standalone-Client} ist
das GUI eines \gloss{Zuwendungsblatt}es eines \gloss{F�rderantrag}s zu sehen.
\myHugeFigure{web.jpg}{Web-Client: Zuewndungsblatt\cite{DegDKElerIAntragsmappe}}{Web-Client}
\myBigFigure{standalone.jpg}{Standalone-Client:
Zuewndungsblatt\cite{DegDKElerIAntragsmappe}}{Standalone-Client} F�r den Aufbau sind nur die Tabelle und
die darunter stehenden Buttons, sowies das Bemerkungsfeld (im Web-Client auf der rechten Seite und im
Standalone-Client in der Mitte) von Bedeutung.\\
Um eine effiziente Arbeitsweise zu erm�glichen, kamen (sekund�re) Anforderungen
wie \emph{Wartbarkeit der Frameworks}, \emph{Abstraktion} und die
\emph{\gloss{Ausdruckskraft}} der Sprachkonstrukte, die zur Entwicklung
verwendet werden.
\newline
\newline
\section{Umsetzung der
Benutzerschnittstellen f�r mehreren Plattformen in der deg (Ist-Zustand)}
F�r die Umsetzung der prim�ren Anforderungen w�re es m�glich gewesen f�r den
Web-Client und dem Standalone-Client separate GUIs mit unterschiedlichen
Frameworks zu entwickeln. Die deg hat jedoch eine L�sung erarbeitet mit der es m�glich ist, ein einmal
beschriebenes GUI auf mehrere Plattformen zu protieren. Das reduziert den
Aufwand zur Entwicklung neuer GUIs, durch eine h�here Abstraktion.
Die L�sung der deg ist das \emph{Multichannel-Framework} (MCF).\\
Innerhalb dieses Frameworks werden die GUIs mittels so genannter
\emph{Pr�sentationsformen} beschreiben. Durch die Verwendung gleicher
Bezeichnungen wird die Ausdruckskraft gesteigert. Die Architektur des
Multichannel-Frameworks ist Abbildung \ref{Abb_MC-Framework} zu entnehmen.
\myBigFigure{mcgrob.png}{Architektur des
Multichannel-Frameworks\cite{jwammc}}{MC-Framework}
Daraus wird deutlich, dass aus Pr�sentationsformen mithilfe der
\emph{Component-Factories} GUIs erzeugt werden, die auf unterschiedlichen
Frameworks basieren\footnote{Hier: Swing, ULC und WingS. Wobei ULC bei der deg
nicht mehr im Einsatz ist.}. Somit ist die deg in der Lage ihre GUIs f�r das
\emph{\gloss{Swing}}-Framework und f�r das \emph{\gloss{WingS}}-Framework mit
nur einer GUI-Beschreibung zu erzeugen.

\section{Probleme des Multichannel-Frameworks}
Beim Einsatz des MCF treten jedoch einige Probleme auf.
Das erste Problem bezieht sich auf die integrierten Framworks (\gls{Swing} und
\gls{WingS}). Beide Frameworks sind verwaltet und werden nicht mehr gewartet.
Um auch in der Zukunft den Anforderungen der Kunden nachkommen zu k�nnen m�ssten
beide Frameworks von den Entwicklern der deg selbst weiterentwickelt werden.
Eine andere M�glichkeit w�re es, wenn die deg andere und modernere Frameworks 
einsetzt um den n�tigen Support der Framework-Entwickler
nutzen zu k�nnen.\\
Das MCF ist in der Theorie so konzipiert, dass es leicht sein
sollte neue Frameworks zu integrieren (siehe Abbildung \ref{Abb_MC-Framework}.
In der Praxis hat sich jedoch das Gegenteil gezeigt. Das
Problem, welches bei der Integration neuer Frameworks aufkommt, ist, dass sich
das MCF sehr stark an Swing orientiert und die GUIs vor allem
vom \gloss{GridBagLayout} stark beeinflusst sind. Ein solches Layout steht nicht
in allen Frameworks zur Verf�gung. Da die Beschreibung �ber ein solches Layout
statt findet und der Aufbau der GUIs unterschiedlicher Frameworks per
Anforderung gleich sein soll, ist es die Verf�gbarkeit eines
\gls{GridBagLayout}s somit eine Voraussetzung f�r die Intergration in das
MCF.
Zusammenfassen sind folgende Probleme des MCF zu nennen:
\begin{itemize}
  \item verwendeten Frameworks sind inaktuell
  \item	Wartbarkeit der Frameworks eingeschr�nkt
  \item Starke Orientierung an Swing
\end{itemize}

\section{Zielsetzung}
Das langfristige Ziel der deg bzgl. des MCF ist es, eine L�sung zu entwickeln,
welche das MCF abl�sen kann. Anzustreben ist eine L�sung, die neben den prim�ren
Anforderungen die sekund�ren Anforderung besser umsetzt als das MCF.\\
In dieser Arbeit wird ein Ansatz unterschucht, bei dem es m�glich ist, die oben
genannten sekund�ren Anforderungen besser umzusetzen. Kern des Ansatzen ist eine
DSL, mit deren Hilfe die UIs beschrieben werden sollen. Eine DSLs kann so
konzipiert werden, dass sie ausreichend abstrakt, erweiterbar und
ausdruckst�rker ist als das MCF. Die prim�ren Anforderungen d�rfen dabei nicht
au�er Acht gelassen werden.\\
Die genaue L�sungsidee mittels DSL, welche in dieser Arbeit verfolgt wird, ist
in Kapitel \ref{L�sungsidee} beschrieben.








