\chapter{Entwicklung des Generators f�r das Generieren von Klassen f�r
das Multichannel-Framework}\label{Entwicklung des Generators f�r das Generieren von Klassen f�r
das Multichannel-Framework}
Alle Umsetzungen die in diesem beschrieben werden, fanden in der 3. Iteration
statt. Von daher wird in den folgenden Kapitel nicht mehr zwischen
Iterationen unterschieden.\\
Ziel ist es mit dem Generator und einem entsprechenden DSL-Skript eine
Exploreransicht zu erststellen, wie sie in profil c/s gel�ufig ist. Abbildung
\ref{Abb_ee} zeigt ein solches GUI. Darin enthalten sind zwei B�ume (auf der
linken Seite) und ein CardPanel (auf der rechten Seite), worin der Inhalt der
Ausgew�hlten Elemente der B�ume angezeigt werden soll. Das Anzeigen des Inhalts
wird jedoch nicht umgesetzt.
\myBigFigure{explorerbaum.jpg}{Einfacher Explorer}{ee}
Die notwendigen DSL-Skripte werden bei der Umsetzung vorgestellt.
\section{Beschreibung der GUI- und IP-Klassen im Multichannel-Framework }
Durch die Beschreibung der GUIs mit der entwickelten DSL, ist es m�glich die
GUI- und die IP-Klassen (siehe Kapitel \ref{Analyse}) zu generieren. Die
GUI-Klasse soll dabei vollst�ndig generiert werden. Zur Umsetzung dessen, werden
Informationen �ber das Layout aus der Layout-Datei ben�tigt. Die Umsetzung
und Einbindung dieser Datei nicht in dieser Arbeit nicht behandelt. Aus
diesem Grund wird in dem Generator ein Layout festgelegt. Die IP-Klasse soll nur
teilweise generiert werden. Somit unterteilt sich das folgende
Kapitel in zwei Abschnitte (GUI-Klasse und IP-Klasse).\\
In den GUI-Klassen der deg sind die UI-Komponenten durch
\emph{Pr�sentationsformen} beschrieben (siehe Kapitel \ref{Analyse}). Alle
UI-Komponenten sind in diesen Klassen global zugreifbar, da sie in
Instanzvariabeln abgelegt werden. Die Instanzvariablen stehen bei der deg am
Ende der Klasse. Die bestehende Struktur der GUI-Klassen nicht ver�ndert
werden soll. Beim Implementieren des Generators muss darauf geachtet werden.
Da die Interaktion von der Pr�sentation getrennt ist, m�ssen zur
Referenzierung von Interaktionen zu UI-Komponenten entsprechende Schl�ssel
vergeben werden.\\
Die IP-Klassen ordnen den UI-Komponenten mit Hilfe dieses Schl�ssels
entsprechende Interaktionen zu. Was genau bei der Interaktion gesehen soll, kann
vom Generator nicht erzeugt werden. Dies muss vom Entwickler nachgepflegt
werden.\\ 
In beiden F�llen handelt es sich in dieser Arbeit um Transformer Generation
(siehe Kapitel \ref{Dom�nenspezifische Sprachen}). In Bezug auf die IP-Klassen
ist auch die Templated Generation (siehe Kapitel \ref{Dom�nenspezifische
Sprachen}) in Erw�gung zu ziehen. Um bei der Art der Generation einheitlich zu
sein, wird auf diese M�glichkeit nicht weiter eingegangen.
\section{Umsetzung des frameworkspezifischen Generators}
\subsection*{GUI-Klasse}
Bei Betrachtung der Pr�sentation aus Abbildung \ref{Abb_ee} k�nnen die
Baumstrukturen auf der linken Seite als eine GUI-Beschreibungen betrachtet
werden, die mit zwei fachlichen Konzepten Assoziiert werden k�nnen. Der obere
Baum kann mit dem fachlichen Konzept des Inhaltsbaumes in Verbindung gebracht
werden und der untere mit dem des Verweisebaums. Da sich die fachlichen Konzepte
erkennen lassen, sollten separate GUI-Beschreibungen erstellt werden (siehe
Listing \ref{LST_Inhaltsbaum} und \ref{LST_Verweisebaum}.
\begin{lstlisting}[{caption = GUI-Beschreibung Inhaltsbaum},{label =
LST_Inhaltsbaum}]
type: INNERCOMPLEX;
Label as: 'Kopfzeile'->text='Inhaltsbaum';
Tree<Inhalt> as: 'Inhaltsbaum';
Structure:'Kopfzeile','Inhaltsbaum';
\end{lstlisting}
\begin{lstlisting}[{caption = GUI-Beschreibung Verweisebaum},{label =
LST_Verweisebaum}]
type: INNERCOMPLEX;
Label as: 'Kopfzeile'->text='Verweisebaum';
Tree<Verweise> as: 'Verweisebaum';
Structure:'Kopfzeile','Verweisebaum';
\end{lstlisting}

Bei der Generierung betrachtet Xtext eine GUI-Beschreibung immer im Ganzen. F�r
jede GUI-Beschreibung soll eine eigene Klasse angelegt werden. Die Klassen
enthalten in diesen beiden F�llen zwei globale Variablen. Dar�ber hinaus
enthalten sie Importe, die am Anfang der beiden Klassen stehen. Um die Importe
an diese Stelle zu schreiben, ist es notwendig die GUI-Beschreibung auf ben�tigte Importe zu
untersuchen und zu diese zu speichern (siehe Listing \ref{LST_Importe}).
\begin{lstlisting}[{caption = Speichern der Importe},{label =
LST_Importe}]
def addImport(String newImport) {
		if (!imports.contains(newImport)) {
			imports.add(newImport)
		}
	}
\end{lstlisting}
 Von daher wird der
generierte Quell-Code nicht sofort in die Datei geschrieben, sondern
zwischengespeichert. So ist es m�glich die Importe an den Anfang der Datei zu
schreiben.\\
Die globalen Variablen werden zusammen mit den Importen generiert und in eine
separate Varialbe angespeichert (siehe Listing \ref{LST_GlobalVars}).
\begin{lstlisting}[{caption = Speichern der globalen Varialben},{label =
LST_GlobalVars}]
def addGlobalVar(String globalVar) {
		if(!globalVars.contains(globalVar))
			globalVars.add(globalVar)
	}
\end{lstlisting}
Dadruch
ist es m�glich diese hinter die generierten Methoden zu schreiben.\\
Zu Beginn einer Generierung muss zwischen den Typen der GUI-Beschreibung
unterschieden werden. Je nachdem wird die Beschreibung als \emph{Window} oder
\emph{Innercomplex} compiliert und ben�tigt entsprechende Importe. In den F�llen
der o.g. B�ume wird eine Innercomplex compiliert. Der Klassenkopf wird wie folgt
erzeugt (siehe Listing \ref{LST_HeadComplex}).
\begin{lstlisting}[{caption = Speichern der globalen Varialben},{label =
LST_GlobalVars}]
	def compileComplex(UIDescription description) '''
		�addImport("import DE.data_experts.jwammc.core.pf.PfPanel;")�
		public class �guiFilename� extends PfPanel{
			public �guiFilename�(){
				super( new BorderLayout() );
				try {
     				init();
     			}
    			catch ( Exception e ) {
     				 e.printStackTrace();
   				 }
			}
		�description.init�
		�genGlobalVars�
		}	
	'''
\end{lstlisting}
Zum Abschluss dieser Methoden werden zwei weitere Methoden aufgerufen. Die erste
(\emph{descrition.init}) generiert die im Konstruktor aufgerufene init-Methode.
Die andere ist f�r die Generierung der globalen Variablen zust�ndig.
\subsection*{IP-Klasse}
