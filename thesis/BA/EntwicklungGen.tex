\chapter{Entwicklung des Generators f�r das Generieren von Klassen f�r
das Multichannel-Framework}\label{Entwicklung des Generators f�r das Generieren von Klassen f�r
das Multichannel-Framework}

Die GUI bei der deg teilt sich in drei Bereiche. Diese Dreiteilung entspricht
dem WAM-Ansatz entnommen. Auf diesen Ansatz wird in Kapitel \ref{Entwicklung des
Generators f�r das Generieren von Klassen f�r das Multichannel-Framework} etwas
genauer eingegangen. Der erste Teil �bernimmt die Beschreibung des Aufbaus
des UIs (GUI-Part). Der zweite Teil enth�lt die Interaktionsformen mit den
Komponenten, welche im GUI-Part deklariert wurden. Dieser Teil wird als
\emph{Interaction Part} (IP) bezeichnet. Der dritte Teil enth�lt den
funktionalen Teil der GUI und wird \emph{Functionally Part} (FP) genannt.\\
Bei den �berlegungen dar�ber, wie die DSL umzusetzen ist, wurde fr�hzeitig
entschieden, dass der FP in der DSL nicht beschrieben wird. Es soll nur
der GUI-Part neben einigen Inhalten aus dem IP beschrieben.\\
\section{Syntax und Semantik f�r die Beschreibung der GUIs im
Multichannel-Framework }
\section{Umsetzung des frameworkspezifischen Generators}