
\newglossaryentry{DSL}
{
  name=DSL,
  description={ist eine Sprache die f�r ein bestimmtes Problem optimiert ist.}
}



\newglossaryentry{Graphical User Interfaces (GUI)}
{
	name=GUI,
	 description={ist die Schnittstelle zwischen dem Benutzer und dem Programm}
}


\newglossaryentry{Usability}
{
	name=Usability,
	description={beschreibt die Nutzerfreundlichkeit einer GUI, sowie auch die
	Nutzerfreundlichkeit einer Software} }

\newglossaryentry{INVEKOS}
{
	name=InVeKoS,
	description={ist die Abk�rzung f�r Integriertes Verwaltungs- und
	Kontrollsystem. Mit einem solchen Sysmten wird im allgemeinen sichergestellt,
	dass die durch den Europ�ischen Garantiefonds f�r die Landwirtschaft finanzierten Ma�nahmen
	ordnungsgem�� umgesetzt wurden. Im speziellen bedeutet dies die Absicherung,
	von Zahlungen, die korrekte behandlung von Unregelm��igkeiten und das wieder
	Einziehen von zu unrecht gezahlter Beitr�ge\cite{europa}} }


\newglossaryentry{Zuwendungsblatt}
{
	name=Zuwendungsblatt,
	description={ist die grafische Dokumentation der Ergebisse des\\
	\gls{Zuwendungs-Berechner}s innerhalb von profil c/s} }


\newglossaryentry{Zuwendungs-Berechner}
{
	name=Zuwendungs-Berechner,
	description={ist ein Werkzeug innerhalb von profil c/s. \emph{Mit diesem
	Werkzeug kann der Sachbearbeiter die Zuwendung, die dem Antragsteller bewilligt werden soll, 
	nach einem standardisierten Verfahren berechnen (siehe Abschnitt "Algorithmen"). 
	Das Ergebnis wird im \gls{Zuwendungsblatt} dokumentiert, das auch sp�ter mit
	demselben Werkzeug angesehen werden kann}\cite{DegDKElerIAntragsmappe}} }
	
	
\newglossaryentry{F�rderantrag}
{
	name=F�rderantrag,
	description={\emph{[\ldots] ist ein Antrag, den der Beg�nstigte einreicht, wenn er
	sich eine Ma�nahme f�rdern lassen m�chte}\cite{pcswikifoerd}}
}

\newglossaryentry{Swing}
{
	name=Swing,
	description={ist ein UI-Framework f�r Java Applikationen\cite{swing}}
}

\newglossaryentry{WingS}
{
	name=wingS,
	description={ist ein Framework f�r die komponentenorientierte
	von Webapplikationen\cite{wings}} }
	

\newglossaryentry{GridBagLayout}
{
	name=GridBagLayout,
	description={ist ein Layout Manager innerhalb von \gls{Swing}, welcher die
	Komponenten horizontal, vertical und entlang der Grundlinie anordnet.
	Dabei m�ssen die Komponenten nicht die gleiche Gr��e haben\cite{gridbag}} }
	
	
	
	
\newglossaryentry{Traditionelle UI-Entwicklung}
{
	name=Traditionelle UI-Entwicklung,
	description={Bei der traditionellen
	UI-Entwicklung wird mit traditionellen
	UI-Toolkits gearbeitet. Bei diesen Toolkits wird Aufbau der GUI 
	genau beschrieben.
	F�r die Interaktion mit den UI-Widgets, werden Listener implementiert, die
	auf andere Events reagieren, die von anderen Widgets erzeugt generiert wurden.
	Events k�nnen zu unterschiedlichen Zeitpunkten generiert werden und es wird
	nicht festgelegt in welcher Reihenfolge sie bei anderen Widgest ankommen.
	\cite{10}}}
	
	
	
	\newglossaryentry{Ausdruckskraft}
{
	name=Ausdruckskraft,
	description={Lesbarbeit und Verst�ndlichkeit}}
	
	

	