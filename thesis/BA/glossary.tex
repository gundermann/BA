


\newglossaryentry{GUI}
{
	name=Graphical User Interface,
	 description={- ist die Bezeichnung f�r die Schnittstelle zwischen dem
	 Benutzer und dem Programm \simplevcite{gui}. Die Kurzform \gui wird in
	 dieser Arbeit aufgrund des allgemeinen Sprachgebrauchs als feminines Nomen
	 verwendet (die \emph{GUI})}, short=GUI,
	 plural=GUIs}


\newglossaryentry{Usability}
{
	name=Usability,
	description={- beschreibt die Nutzerfreundlichkeit einer Software
	\vcite{dirnbauer2000usability}{10}}}

\newglossaryentry{InVeKoS}
{
	name=InVeKoS,
	description={- bezeichnet das \emph{Integriertes Verwaltungs- und
	Kontrollsystem}. Mit einem solchen Systemen wird im Allgemeinen sichergestellt,
	dass die durch den Europ�ischen Garantiefonds f�r die Landwirtschaft finanzierten Ma�nahmen
	ordnungsgem�� umgesetzt wurden. Im Speziellen bedeutet dies die Absicherung
	von Zahlungen, die korrekte Behandlung von Unregelm��igkeiten und die
	R�ckforderung von zu unrecht gezahlten Beitr�gen \simplevcite{europa}} }


\newglossaryentry{Zuwendungsblatt}
{
	name=Zuwendungsblatt,
	description={- ist die grafische Dokumentation der Ergebnisse des\\
	\glossLink{Zuwendungs-Berechner}{Zuwendungsberechners} innerhalb von \pcs
	\simplevcite{DegDKElerIAntragsmappe}}}


\newglossaryentry{Zuwendungs-Berechner}
{
	name=Zuwendungsberechner,
	description={- ist ein \emph{Werkzeug} innerhalb von \emph{profil c/s}.
	\simpledcite{Mit diesem Werkzeug kann der Sachbearbeiter die Zuwendung, die dem Antragsteller bewilligt werden soll, 
	nach einem standardisierten Verfahren berechnen [\ldots] . 
	Das Ergebnis wird im \gls{Zuwendungsblatt} dokumentiert, das auch sp�ter mit
	demselben Werkzeug angesehen werden kann}{DegDKElerIAntragsmappe}}}
	
	
\newglossaryentry{F�rderantrag}
{
	name=F�rderantrag,
	description={- \simpledcite{[\ldots] ist ein Antrag, den der
	Beg�nstigte einreicht, wenn er sich eine Ma�nahme f�rdern lassen m�chte}{pcswikifoerd}}
}

\newglossaryentry{Swing}
{
	name=Swing,
	description={- ist ein \emph{GUI-Framework} f�r \emph{Java-Applikationen}
	\simplevcite{swing}} }

\newglossaryentry{wingS}
{
	name=wingS,
	description={- ist ein Framework f�r die komponentenorientierte Entwicklung
	von Webapplikationen \simplevcite{wings}}}
	

\newglossaryentry{GridBagLayout}
{
	name=GridBagLayout,
	description={- ist ein Layoutmanager innerhalb von \gls{Swing}, welcher die
	Komponenten horizontal, vertikal und entlang der Grundlinie anordnet.
	Dabei m�ssen die Komponenten nicht die gleiche Gr��e haben
	\simplevcite{gridbag}}}
	
	
	
	
\newglossaryentry{Traditionlle GUI-Entwicklung}
{
	name=Traditionelle GUI-Entwicklung,
	description={- beschreibt die 
	\emph{GUI-Entwicklung} unter Verwendung von \emph{traditionellen
	GUI-Toolkits}. Bei diesen Toolkits wird der Aufbau der \gui 
	genau beschrieben.
	F�r die Interaktion mit den \emph{GUI-Komponenten}, werden Listener
	implementiert, die auf andere Events reagieren, die wiederum von anderen Komponenten erzeugt
	wurden.
	Events k�nnen zu unterschiedlichen Zeitpunkten generiert werden, wobei die
	Reihenfolge, wie sie bei anderen Komponenten ankommen, nicht festgelegt ist
	\simplevcite{10}}}
	
	
	
	
\newglossaryentry{Turing-Maschine}
{
	name=Turing-Maschine,
	description={- ist ein Automatenmodell, welches vom Alan M. Turing im Jahr 1936
	vorgestellt wurde. Die Turing-Maschine abstrahiert die generelle
	Arbeitsweise heutiger Rechner \vcite{hedtstueck}{145ff}}}
	
			
\newglossaryentry{Dom�ne}
{
	name=Dom�ne,
	description={oder Anwendungsdom�ne - beschreibt ein abgegrenztes Wissens- oder
	Interessengebiet \vcite{softarch}{170}}}
	
	
\newglossaryentry{DSL-Umgebung}
{
	name=DSL-Umgebung,
	description={- beinhaltet den Parser, den Lexer und die
	Verarbeitungslogik \vcite{ghosh}{211}}}

\newglossaryentry{Zielumgebung}
{
	name=Zielumgebung,
	description={einer \emph{DSL} - beschreibt die Infrastruktur, in der
	die generierten Artefakte integriert werden k�nnen \vcite{voelter}{26}}}
	
	
\newglossaryentry{Multiselection-Komponente}
{
	name=Multiselection-Komponente,
	description={- stellt in \pcs ein Werkzeug
	zur Auswahl mehrerer Objekte dar. Es
	wird zwischen zwei Containern unterschieden. Der Container auf der linken
	Seite enth�lt der die Objekt, die zur Auswahl stehen und der 
	Container auf der rechten Seite enth�lt die Objekte, die bereits
	ausgew�hlt sind. Mit den in der Mitte befindlichen Schaltfl�chen ist es
	m�glich, die Objekte von einem Container in den anderen zu navigieren
	 }}
	
	
	
\newglossaryentry{Top-Down Parser}
{
	name=Top-Down Parser,
	description={- erzeugen den \emph{Parse Tree}, ausgehend von der Wurzel. Im
	Gegensatz dazu stehen \emph{Bottom-Up Parser}, welche den \emph{Parse Tree} von den Bl�tter aus
erzeugen
\vcite{ghosh}{225}}}


\newglossaryentry{Inhaltsbaum}
{
	name=Inhaltsbaum,
	description={- enth�lt die Dokumente die in
	dieser Antragsmappe vorliegen}}
	
\newglossaryentry{Verweisebaum}
{
	name=Verweisebaum,
	description={- enth�lt Verweise auf
	Dokumente, die mit der Antragsmappe in Verbindung stehen}}
