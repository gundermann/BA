\chapter{Zusammenfassung und Ausblick}\label{Zusammenfassung und Ausblick}
Durch die Generierung der GUI-, IP- und FP-Klassen muss die Routinearbeit R1
(siehe Kapitel \ref{Analyse}) nicht mehr durchgef�hrt werden. Die Generierung
von Tabellen in Hinblick auf Vermeidung der Routinearbeit R2 (siehe Kapitel
\ref{Analyse}) konnte aus Zeitgr�nden nicht mehr umgesetzt werden. Es ist
jedoch auf Basis der gezeigten Generierungen davon auszugehen, dass den
Entwicklern auch diese Arbeit vom Generator abgenommen werden kann. Daf�r bedarf
es jedoch die Umsetzung und Einbindung der Layout-Datei. \\
Bez�glich der Sprache kann die Anforderung \emph{weniger LOC zur Beschreibung
von UIs} (siehe Kapitel \ref{GUI-DSL}) erst an dieser Stelle als umgesetzt
betrachtet werden. Zum Vergleich sind die generierten Klassen und die
DSL-Skripte auf deren Basis jene erzeugt wurden im Anhang \ref{AppGeneriert} zu
finden. Die DSL-Skripte umfassen dabei lediglich 13 Codezeilen. Die
daraus generierten Klassen umfassen mehr als 200 LOC.\\
Test und Ausf�hrung

Einbettung in den Entwicklungsprozess

protected regions

Generierung Eclipse Editor