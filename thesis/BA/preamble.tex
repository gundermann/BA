
\usepackage[german]{babel}
\usepackage[ansinew]{inputenc}
\usepackage[T1]{fontenc} %Umlaute
\usepackage{lmodern}
\usepackage[top=2cm, left=3cm, bottom=2cm, right=4cm]{geometry}
\usepackage{fancybox, graphicx}
\usepackage{float}
\usepackage{listings} 
\usepackage{multirow}
\usepackage{color}		 % f�r Farben im allgemeinen
\usepackage{colortbl}
\usepackage{cite}
\usepackage{bibgerm}
\usepackage{palatino}
\usepackage{chngpage}
\usepackage{chngcntr}
\counterwithout{figure}{chapter}
\counterwithout{table}{chapter}
\renewcommand{\floatpagefraction}{0.85}

\usepackage{fancyhdr}
\pagestyle{fancy}

\definecolor{rot}{rgb}{1,0.3,0}
\definecolor{gelb}{rgb}{1,1,0}
\definecolor{gruen}{rgb}{0,1,0.4}


\fancyhf{}
\fancyhead[RE]{\slshape \nouppercase{\leftmark}}    % Even page header: "page   chapter"
\fancyhead[LO]{\slshape \nouppercase{\rightmark}}   % Odd  page header: "section   page"
\fancyhead[RO,LE]{\bfseries \thepage} 
\renewcommand{\headrulewidth}{1pt}    % Underline headers
\renewcommand{\footrulewidth}{0pt}    

\fancypagestyle{plain}{               % No chapter+section on chapter start pages
\fancyhf{}
\fancyhead[RO,LE]{\bfseries \thepage}
\renewcommand{\headrulewidth}{1pt}
\renewcommand{\footrulewidth}{0pt}
}

% Left headings: "1  INTRODUCTION"
%\renewcommand{\chaptermark}[1]{%
%\markboth{\thechapter\ \ \ \ #1}{}}

% Right headings: "1.1  Basics"
\renewcommand{\sectionmark}[1]{%
\markright{\thesection\ \ \ \ #1}{}}

%\lstset{language=java} 
\lstset{basicstyle=\scriptsize}
\lstset{numbers=left, numberstyle=\tiny, numbersep=2pt, breaklines=true} 
\lstset{ numberbychapter=false}
\graphicspath{{Bilder/}}
\newcommand{\listoflolentryname}{\lstlistingname} 
\usepackage{float}
%\newcommand{\fullcite}{\citep} %for "Author [1980]"
\usepackage[pdftex,plainpages=false,pdfpagelabels]{hyperref}
\usepackage[nonumberlist]{glossaries}

\makeglossaries
\loadglsentries{glossary}


% --- Farbdefinitionen ----------------------------------------
\definecolor{rot}{rgb}{1,0.3,0}
\definecolor{gelb}{rgb}{1,1,0}
\definecolor{gruen}{rgb}{0,1,0.4}
\definecolor{darkblue}{rgb}{0.2,0.3,1}
\definecolor{lightblue}{rgb}{0.6,0.7,1}
\definecolor{white}{rgb}{1,1,1}

\linespread{1.5}

\bibliographystyle{geralpha}

\renewcommand{\floatpagefraction}{0.85}

\usepackage[figuresright]{rotating}
\usepackage{geometry}
\geometry{a4paper,left=40mm,right=30mm} 

\newlength{\fullwidth} % Width of text plus margin notes
\setlength{\fullwidth}{\textwidth}


%----------------------------------------------------------------------------------
% \myBigFigure	[ LABEL_PREFIX (optional) ]
%				{ FILENAME (without extension) }
%				{ CAPTION TEXT }
%				{ SHORT VERSION OF CAPTION TEXT }
%
%Bild wird in kompletter Breite gesetzt
%picture using full width of the page
\newcommand{\myBigFigure}[4][Abb]
{
\begin{figure}[H]
	\center{
	\begin{minipage}{\fullwidth}
		\includegraphics[width= \fullwidth]{#2}
		\caption{#3}
		\label{#1_#4}
	\end{minipage}
	}
\end{figure}
}


%----------------------------------------------------------------------------------
% \myBigFigure	[ LABEL_PREFIX (optional) ]
%				{ FILENAME (without extension) }
%				{ CAPTION TEXT }
%				{ SHORT VERSION OF CAPTION TEXT }
%
%Bild wird in kompletter Breite gesetzt
%picture using full width of the page
\newcommand{\myBigFigureCited}[5][Abb]
{
\begin{figure}[H]
	\center{
	\begin{minipage}{\fullwidth}
		\includegraphics[width= \fullwidth]{#2}
		\caption[#3]{#3#4}
		\label{#1_#5}
	\end{minipage}
	}
\end{figure}
}


%----------------------------------------------------------------------------------
% \dcite	{ Text }
%				{ source }
%				{ page }
%
%Direktes Zitat
\newcommand{\dcite}[3]
{
\emph{\glqq#1\grqq}\cite[S.#3]{#2}}

%----------------------------------------------------------------------------------
% \dcite	{ Text }
%				{ source }
%				{ page }
%
%Direktes Zitat
\newcommand{\simpledcite}[2]
{
\emph{\glqq#1\grqq}\cite{#2}}

%----------------------------------------------------------------------------------
% \vcite	
%				{ source }
%				{ page }
%
%Direktes Zitat
\newcommand{\vcite}[2]
{
(vgl. \cite[S.#2]{#1})}


%----------------------------------------------------------------------------------
% \vcite	
%				{ source }
%				{ page }
%
%Direktes Zitat
\newcommand{\simplevcite}[1]
{
(vgl. \cite{#1})}

%----------------------------------------------------------------------------------
% \myHUGEFigure	[ LABEL_PREFIX (optional) ]
%				{ FILENAME (without extension) }
%				{ CAPTION TEXT }
%				{ SHORT VERSION OF CAPTION TEXT }
%
%Bild wird rotiert und quer in kompletter Breite gesetzt
%landscape picture using the full width of the rotated page
\newcommand{\myHugeFigure}[4][Abb]
{
\begin{sidewaysfigure}[H]
	
		\includegraphics[width= \textheight]{#2}
		\caption{#3}
		\label{#1_#4}
	
\end{sidewaysfigure}
}


%----------------------------------------------------------------------------------
% \myHUGEFigure	[ LABEL_PREFIX (optional) ]
%				{ FILENAME (without extension) }
%				{ CAPTION TEXT }
%				{ SHORT VERSION OF CAPTION TEXT }
%
%Bild wird rotiert und quer in kompletter Breite gesetzt
%landscape picture using the full width of the rotated page
\newcommand{\myHugeFigureCited}[5][Abb]
{
\begin{sidewaysfigure}[t!bp]
	
		\includegraphics[width= \textheight]{#2}
		\caption[#3]{#3#4}
		\label{#1_#5}
	
\end{sidewaysfigure}
}


\newcommand{\gloss}[1]{\emph{\gls{#1}}}

\newcommand{\glossLink}[2]{
%#1\footnote{Siehe Glossar: 
\emph{\glslink{#1}{#2}}
%}
}

\newcommand{\Ks}{\emph{Basiskomponenten} }

\newcommand{\K}{\emph{Basiskomponente} }
\newcommand{\knks}{\emph{komplexen Komponenten} }

\newcommand{\kks}{\emph{komplexe Komponenten} }

\newcommand{\guis}{\emph{\glspl{GUI}} }

\newcommand{\gui}{\emph{\acrshort{GUI}} }

\newcommand{\g}{\emph{GUI-DSL} }

\renewcommand{\deg}{\emph{data experts GmbH} }

\newcommand{\gk}{\emph{GUI-Komponente} }
\newcommand{\gks}{\emph{GUI-Kompo\-nen\-ten} }

\newcommand{\MCF}{\emph{MCF} }
\newcommand{\pcs}{\emph{profil c/s} }

\newcommand{\DSL}{\emph{DSL} }

