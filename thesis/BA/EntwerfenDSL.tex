\chapter{Entwicklung einer DSL zur Beschreibung der GUI in profil c/s}\label{Entwicklung einer DSL zur Beschreibung der GUI in profil c/s}
\section{Analyse der Metadaten der GUI}
�berlegungen bzgl. des Aufbaus der UIs gingen dahin, dass sich die DSL an dem
Komponentenmodel von RCP4 orientieren soll. Das bedeutet, dass es Meta-Ebene
gibt, die den groben Aufbau der GUI beschreibt und eine Implementierungs-Ebene,
welche spezifische Komponenten innerhalb der GUI umsetzen soll. Beim Aufbau
der GUI wurde Anfangs zwischen zwei Typen unterschieden (siehe semantisches
Model Type). Jede GUI enth�lt eine bestimmte Anzahl von Bereichen, denen andere
UI-Komponenten zugeordnet werden k�nnen (siehe semantisches Model Areas und
areacount).
Weiterhin k�nnen von einer GUI-Beschreibung andere GUI-Beschreibungen verwendet
werden (siehe semantisches Model Use). Die erwendeten Beschreibungen k�nnen
jedoch nicht erweitert werden. Den Kern der GUI-Beschreibung bildet jedoch die
Kompnentendefinition (siehe semantisches Model componentDefinition). Dort werden
einzelne Komponeten der GUI durch die Meta-Daten beschrieben.\\
Bezogen auf die tivialen Komponenten des UIs war f�r die ersten Versuche nur die
Beschreibung eines Textes wichtig. Im Falle eines Buttons oder eines Labels
beschreibt dieser die Aufschrift der Komponente. Weiterhin war es f�r die
Zuweisung zu den entsprechenden Bereich wichtig, dass diese Komponenten
innerhalb der Datei referenziert werden k�nnen. Das wurde durch den Titel
umgesetzt, der an der Komponente gesetzt werden muss. Die Festlegungen die bis
hierhin getroffen wurden, umfassten auch die Beschreibung von Interaktionen.
Daher muss an jeder trivialen Komponente ein Interaktionstyp beschrieben
werden k�nnen. Zusammenfassend ergaben sich bis hierhin folgende Meta-Daten
der trivialen Komponenten.
\begin{itemize}
  \item Typ
  \item Titel
  \item Text
  \item Interaktionsform
\end{itemize}
Bei den komplexen Komponenten musste vor der Analyse eine weitere Entscheidung
getroffen werden, da es zwei Implementierungsm�glichkeiten gab, die in Frage
kamen. 
\begin{itemize}
  \item Die komplexen Komponenten wedern als in der Grammatik als
  Komponenten-Definition aufgenommen.
  \item Die komplexen Komponenten werden als verwendete Komponenten aufgenommen.
\end{itemize}
Da mit der zweiten Variante keine Erweiterungsm�glichkeiten des Inputs der
genutzen Komponenten m�glich ist, ist die erste Variante die bessere Wahl.
Au�erdem wird somit das Fehlerrisiko in der DSL eingeschr�nkt, da die Zuweisung
verwendeter Komponenten mittels Strings get�tigt wird. 











Die GUI bei der deg teilt sich in drei Bereiche. Diese Dreiteilung entspricht
dem WAM-Ansatz entnommen. Auf diesen Ansatz wird in Kapitel \ref{Entwicklung des
Generators f�r das Generieren von Klassen f�r das Multichannel-Framework} etwas
genauer eingegangen. Der erste Teil �bernimmt die Beschreibung des Aufbaus
des UIs (GUI-Part). Der zweite Teil enth�lt die Interaktionsformen mit den
Komponenten, welche im GUI-Part deklariert wurden. Dieser Teil wird als
\emph{Interaction Part} (IP) bezeichnet. Der dritte Teil enth�lt den
funktionalen Teil der GUI und wird \emph{Functionally Part} (FP) genannt.\\
Bei den �berlegungen dar�ber, wie die DSL umzusetzen ist, wurde fr�hzeitig
entschieden, dass der FP in der DSL nicht beschrieben wird. Es soll nur
der GUI-Part neben einigen Inhalten aus dem IP beschrieben.\\

\section{Semantisches Model}
Das Semantische Model in den ersten Wochen beinhaltete 
\section{Syntax}

