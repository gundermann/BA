\chapter{Entwicklung einer DSL zur Beschreibung der GUI in profil c/s}\label{Entwicklung einer DSL zur Beschreibung der GUI in profil c/s}
Die Entwicklung der DSL und die Entwicklung des Generators (siehe Kapitel
\ref{Entwicklung des Generators f�r das Generieren von Klassen f�r das
Multichannel-Framework}) gliedern sich in mehrere Versionen. In jeder Version
sind Ideen umgesetzt, die nach der Fertigstellung einer Version in der deg
begutachtet wurden und wenn n�tig abgewandelt oder ver�ndert.
\section{Analyse der Metadaten der GUI}
\subsection*{Version 1}
�berlegungen bzgl. des Aufbaus der UIs gingen dahin, dass sich die DSL an dem
Komponentenmodel von RCP4 orientieren soll. Das bedeutet, dass es Meta-Ebene
gibt, die den groben Aufbau der GUI beschreibt und eine Implementierungs-Ebene,
welche spezifische Komponenten innerhalb der GUI umsetzen soll. Beim Aufbau
der GUI wurde zwischen zwei Typen unterschieden (siehe semantisches
Model Type). F�r die Beschreibung des Aufbaus enth�lt jede beschriebene GUI
eine bestimmte Anzahl von Bereichen (Area), denen genau eine andere
UI-Komponent zugeordnet werden kann (siehe semantisches Model Areas und
areacount).
Weiterhin k�nnen von einer GUI-Beschreibung andere GUI-Beschreibungen verwendet
werden (siehe semantisches Model Use). Die verwendeten GUI-Beschreibungen k�nnen
jedoch nicht erweitert werden. Den Kern der GUI-Beschreibung jedoch die
Kompnentendefinition (siehe semantisches Model componentDefinition). Dort werden
einzelne Komponeten der GUI durch die Meta-Daten beschrieben.\\
Bezogen auf die tivialen Komponenten des UIs die
Beschreibung eines Textes wichtig. Im Falle eines Buttons oder eines Labels
(andere triviale Komponenten sind in dieser Version nicht umgesetzt) beschreibt
dieser die Aufschrift der Komponente.
Weiterhin war es f�r die Zuweisung zu den entsprechenden Bereich wichtig, dass diese Komponenten
innerhalb der Datei referenziert werden k�nnen. Das wurde durch den Titel
umgesetzt, der f�r jede Komponente definiert werden muss. An den trivialen
Komponenten k�nnen daruber hinaus Interaktionen beschrieben werden. Hierzu
ist ein Interaktionstyp n�tig. Eine einfacher Klick auf die Komponente ist
der einzige Interaktionstyp in dieser Version. An dieser Interaktion k�nnen
ebenso Aktionen definiert werden, die Auswirkungen auf andere Komponenten haben.
Zusammenfassend ergeben sich folgende Meta-Daten der trivialen
Komponenten.
\begin{itemize}
  \item Typ
  \item Titel
  \item Text
  \item Interaktion
\end{itemize}
Die Interaktion ben�tigt folgende Attributen, die beschrieben werden m�ssen.
\begin{itemize}
  \item Titel
  \item Interaktionstyp
  \item Aktion
\end{itemize}
Die Aktion ben�tigt einen \emph{ActionType}, das \emph{Element} auf das sich
die Interaktion auswirken soll und die Ver�nderung der Attribute des
entsprechenden Elements (\emph{Properties}.\\
Die komplexen Komponenten m�ssen f�r jedes verwendete UI-Framework
implementiert werden. Das hat zur Folge, dass die Implementierung dieser
Komponenten nicht so stark abstrahiert wird, dass sie nur einmal
ebtwickelt werden m�ssen. Damit wird jedoch auch verhindert, dass die
Entwickler, die bzgl. der GUI nur mit der DSL arbeiten, eigene komplexe
Komponenten entwerfen, deren Wiederverwendungsgrad nierdirger ist, als wenn
diese Komponenten nach ausreichender Evaluation an einer zentralen Stelle
implementiert und bereitgestellt werden.\\
Komplexe Komponenten werden wie die trivialen Komponenten in einer
Komponentendefinition beschrieben. Dazu wird nach der Implementierung der
Komponente f�r jedes Framework ein neues Schl�sselwort f�r eine
Komponentendefinition eingebaut. Jede komplexe Komponente ben�tigt wiederum
einen Titel um referenziert zu werden. In dieser Version ist eine
\gloss{Multiselection-Komponente} umgesetzt. Diese Komponente ist generisch
implementiert. Der generische Typ kann in der DSL an dem Schl�sselwort
\emph{InputType} beschrieben werden. Die Werte, die in dieser Komponente
selektiert werden k�nnen, werden �ber das Schl�sselwort \emph{selectableValues}
gesetzt und die Werte, die selektiert sind am Schl�sselwort
\emph{selectedValues}.

\subsection*{Version 2}
\section{Semantisches Model}
\subsection*{Version 1}
Das Semantische Model in den ersten Wochen beinhaltete 
\section{Syntax}
\subsection*{Version 1}

