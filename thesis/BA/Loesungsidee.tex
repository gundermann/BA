\chapter{Entwicklung einer L�sungsidee}\label{Entwicklung einer L�sungsidee}
\section{Allgemeine Beschreibung der L�sungsidee}
Aufgrund der nur schwer machbaren Integration neuer Frameworks in das
bestehende Multichannel-Framework und der Tatsache, dass die derzeit genutzten
Frameworks (\gls{Swing} und \gls{WingS}) veraltet sind, wird ein neuer Ansatz f�r die
Umsetzung von GUIs auf unterschiedlichen Plattformen gesucht.
\myBigFigure{neuerAnsatz.png}{DSL-Ansatz f�r gleich GUIs auf
unterschiedlichen Plattformen}{neuerAnsatz}\\
Der neue Ansatz basiert auf der folgenden Idee. Die GUIs sollen weiterhin nur
einmal beschrieben werden sollen. Diese Beschreibung soll �ber eine DSL
erfolgen und sich nicht an bestehende Frameworks orientieren. Grund daf�r ist,
dass ansonsten die Gefahr besteht, dass langfristig betrachtet mit diesem Ansatz
das gleiche Problem auftirtt wie beim Multichannel-Framework. Aus der
Beschreibung der GUIs wird ein Generator speziellen Quellcode erzeugen, der sich
auf entsprechenden Plattformen ausf�hren l�sst. F�r jedes eingesetzte Framework
muss somit ein eigener Generator entwickelt werden. Abbildung \ref{Abb_neuerAnsatz}
bildet die aus dieser Idee resultierende Architektur ab\footnote{Hier: Vaadin als
Web-Framework, JavaFX als Framework f�r den Standalone-Client und
Android als Repr�sentant f�r einen m�glichen Mobile-Client}.
\section{Architektur}
\section{Vorteile gegen�ber dem Multichannel-Framework}