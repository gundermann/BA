\chapter{Entwicklung einer L�sungsidee}\label{Entwicklung einer L�sungsidee}
\section{Allgemeine Beschreibung der L�sungsidee}
Eine L�sungsidee f�r die in Kapitel \ref{Probleme} beschriebenen Probleme wurde
im Kapitel \ref{Ziel} bereits angedeutet. Es geht um die Nutzung einer
DSL zur Beschreibung von GUIs. Diese GUIs sollen durch die DSL so beschrieben
werden, dass sie in der Dom�ne von profil c/s f�r unterschiedliche UI-frameworks 
genutzt werden kann. Damit kann das MCF langfristig betrachtet abgel�st werden.
\section{Architektur}
In diesem L�sungsansatz ist die DSL der Ausgangspunkt. Die abstrakte
Beschreibung der GUI wird �ber die DSL vorgenommen. Ein Generator
kann nach dem parsen f�r diese Beschreibung der GUI frameworkspezifischen Code
generieren. Somit ist die Integration neuer Frameworks an die Implementierung
eines spezifischen Generators gekoppelt. Abbildung \ref{Abb_neuerAnsatz} zeigt die
Architektur f�r diesen Ansatz auf. Dabei wurden exemplarisch drei
unterschiedliche Generatoren f�r unterschiedliche Frameworks mit aufgenommen.
\myBigFigure{neuerAnsatz.png}{DSL-Ansatz f�r gleich GUIs auf
unterschiedlichen Plattformen}{neuerAnsatz}\\
Somit wird erreicht, dass die GUI weiterhin nur einmal beschrieben werden muss.

\section{Vorteile gegen�ber dem Multichannel-Framework}
Wie in Kapitel \ref{Probleme} erl�utert, wei�t das MCF einige Probleme auf. Mit
dem neuen Ansatz kann das Problem der inaktuellen Frameworks und das Problem der
starken Orientiertung an Swing (oder an ein anderes Framework) beseitigt werden.
Eine DSL sollte sich nicht an Besonderheiten bestehender
Frameworks orientieren, sondern an dem Dom�nenproblem. \cite[S.15]{mdsdvoelter}
Von daher ist sichergestellt, dass die Integration von unterschiedlichen Frameworks gleicher Ma�en gut funktioniert.\\
Ein weiterer Vorteil ist, dass durch die wegfallende Orientiertung an Swing auch
die Beschreibungsform ausdrucksst�rker wird. Grund daf�r ist, dass die
syntaktischen Strukten, die in Swing vorhanden sind, nicht mehr ben�tigt
werden.\\
Dazu kommt, dass fachliche Konzepte zur Beschreibung der UIs benutzt werden
k�nnen und die Umsetzung auf technischer Ebene von den Generatoren �bernommen
wird.
