\chapter{Entwicklung einer L�sungsidee}\label{Entwicklung einer L�sungsidee}
\section{Allgemeine Beschreibung der L�sungsidee}
Eine L�sungsidee f�r die in Kapitel \ref{Probleme} beschriebenen Probleme wurde
im Kapitel \ref{Ziel} bereits angedeutet. Kern dieser Idee ist, eine
DSL zur Beschreibung von GUIs zu nutzen. Diese GUIs sollen so beschrieben
werden, dass sie in der Dom�ne von profil c/s f�r unterschiedliche
UI-Frameworks genutzt werden k�nnen. Diese Beschreibung soll weiterhin nur
einmal stattfinden. Der Quell-Code, welcher das GUI im entsprechenden Framework
darstellt, wird frameworkspezifisch aus der GUI-Beschreibung generiert.
Langfristig betrachtet k�nnte das MCF damit abgel�st werden.
\section{Architektur}
Die, diesem L�sungsansatz zugrunde liegende, DSL wird f�r die abstrakte
Beschreibung der GUI verwendet. Somit ist gew�hrleistet, dass die GUI weiterhin
nur einmal beschrieben werden muss. 
Wie in Abschnitt 4.1 beschrieben, wird der Quell-Code der GUI
frameworkspezifisch, mit Hilfe eines speziellen Generators, erzeugt. Dadruch
ist die Integration neuer Frameworks an die Implementierung eines spezifischen
Generators gekoppelt. Abbildung \ref{Abb_neuerAnsatz} zeigt die Architektur f�r diesen Ansatz auf. Dabei wurden exemplarisch drei unterschiedliche Generatoren f�r unterschiedliche Frameworks mit aufgenommen.
\myBigFigure{neuerAnsatz.png}{DSL-Ansatz f�r gleich GUIs auf
unterschiedlichen Plattformen}{neuerAnsatz}

\section{Vorteile gegen�ber dem Multichannel-Framework}
Wie in Kapitel \ref{Probleme} erl�utert, weist das MCF einige Probleme auf. Mit
der vorgestellten L�sung kann das Problem der inaktuellen
Frameworks und das Problem der starken Orientiertung an Swing (oder an ein anderes Framework) beseitigt werden.
Eine DSL sollte sich nicht an Besonderheiten bestehender
Frameworks orientieren, sondern an dem Dom�nenproblem.
(vgl. \cite[S.15]{mdsdvoelter}) Von daher sollte bei korrekter Umsetzung
sichergestellt sein, dass die Integration von unterschiedlichen Frameworks gleicher Ma�en gut funktioniert.\\
Ein weiterer Vorteil ist, dass durch die wegfallende Orientiertung an Swing auch
die Beschreibungsform ausdrucksst�rker wird. Grund daf�r ist, dass die
syntaktischen Strukten, die in Swing vorhanden sind, nicht mehr ben�tigt
werden und die DSL auf einer h�heren Abstraktionsebene konzipiert werden kann.
Das erweitern der DSL um fachliche Konzepte aus profil c/s sollte somit
erm�glicht werden. Mit diesen fachlichen Konzepten wird eine Ann�herung an das
Model-Driven Development (siehen Kapitel \ref{mdsd}) erreicht.\\
Dar�ber hinaus werden dem Entwickler durch die Codegenerierung die in Kapitel
\ref{Probleme} einige beschriebenen fehleranf�lligen Routinearbeiten abgenommen.
