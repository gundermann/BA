\chapter{Entwicklung einer L�sungsidee}\label{Entwicklung einer L�sungsidee}
\section{Allgemeine Beschreibung der L�sungsidee}\label{AllgBeschreibL�sung}
Eine L�sungsidee f�r die in Kapitel \ref{Probleme}
beschriebenen Probleme wurde im Kapitel \ref{Ziel} bereits angedeutet. Kern dieser Idee
ist, die im vorherigen Kapitel angesprochene \g zur Beschreibung von
\guis zu nutzen.
Diese \guis sollen so beschrieben werden, dass sie in der Dom�ne von \pcs
f�r unterschiedliche \emph{GUI-Frameworks} genutzt werden k�nnen. Diese
Beschreibung soll weiterhin nur einmal stattfinden. Der Quellcode, welcher die \gui im entsprechenden
Framework darstellt, wird frameworkspezifisch aus der
\emph{GUI-Beschreibung} generiert.
Langfristig betrachtet k�nnte das \MCF damit abgel�st werden.\\
Die Anforderungen f�r die \g wurden bereits beschrieben. Diese
sollen so weit wie m�glich im Prototypen, welcher im Zuge dieser Arbeit entwickelt wird,
umgesetzt werden. Der Prototyp soll
Quellcode erzeugen, der eine \gui mit den syntaktischen Strukturen des \MCF
beschreibt. Somit kann gepr�ft werden, ob sich der generierte Quellcode in
\pcs einbinden l�sst.
\section{Konzept}
Die \g  wird f�r die abstrakte
Beschreibung der \gui verwendet. Somit ist gew�hrleistet, dass die \gui weiterhin
nur einmal beschrieben werden muss. 
Wie in Abschnitt \ref{AllgBeschreibL�sung} beschrieben, wird der Quellcode zur
Darstellung der \gui mit Hilfe eines speziellen
\emph{Generators} erzeugt. Daraus folgt, dass die Integration eines neuen
Frameworks (siehe Anforderung \emph{\hyperref[AA2]{AA2}}) an die Implementierung eines
spezifischen \emph{Generators} gekoppelt ist.
In Abbildung \ref{Abb_neuerAnsatz} wird das grundlegende Konzept des
Ansatzes schematisch dargestellt. Dabei wurden exemplarisch drei
unterschiedliche \emph{Generatoren} f�r bestimmte Frameworks verwendet.
\myBigFigure{neuerAnsatz.png}{Grundlegendes Konzept}{neuerAnsatz}
\noindent
Der Prototyp wird aus einem \emph{GUI-Skript} den Quellcode erzeugen, welcher in
das \MCF eingebunden werden kann. Somit l�sst sich pr�fen, ob die \g f�r
existierende \guis von \pcs gen�gt. In Anlehnung an Abbildung
\ref{Abb_mdsd} in Kapitel \ref{mdsd} wird zur Erzeugung des Quellcodes f�r das
\MCF auch individueller Quellcode von N�ten sein.
In Abbildung \ref{Abb_genMCFCode} wird diese grundlegende Idee schematisch
aufgezeigt.
\myBigFigure{genMCFCode.jpg}{Grundlegende Idee f�r den Prototypen}{genMCFCode}
