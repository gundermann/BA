\chapter{Entwicklung einer L�sungsidee}\label{Entwicklung einer L�sungsidee}
\section{Allgemeine Beschreibung der L�sungsidee}
Eine L�sungsidee f�r die in Kapitel \ref{Probleme} beschriebenen Probleme wurde
im Kapitel \ref{Ziel} bereits angedeutet. Es geht um die Nutzung einer
DSL zur Beschreibung von GUIs. Diese GUIs sollen so beschrieben
werden, dass sie in der Dom�ne von profil c/s f�r unterschiedliche
UI-Frameworks genutzt werden k�nnen. Die GUIs werden dabei weiterhin nur
ein mal beschrieben. Der Code, welcher das GUI im entsprechenden Framework
darstellt wird frameworkspezifisch als der GUI-Beschreibung generiert.
Langfristig betrachtet kann das MCF damit abgel�st werden.
\section{Architektur}
In diesem L�sungsansatz ist die DSL der Ausgangspunkt, mit deren Hilfe eine
abstrakte Beschreibung der GUI vorgenommen wird. Somit ist gew�hrleistet,
dass die GUI weiterhin nur einmal beschrieben werden muss. Ein Generator kann
nach dem parsen f�r diese Beschreibung der GUI, frameworkspezifischen Code
generieren. Somit ist die Integration neuer Frameworks an die Implementierung
eines spezifischen Generators gekoppelt. Abbildung \ref{Abb_neuerAnsatz} zeigt
die Architektur f�r diesen Ansatz auf. Dabei wurden exemplarisch drei
unterschiedliche Generatoren f�r unterschiedliche Frameworks mit aufgenommen.
\myBigFigure{neuerAnsatz.png}{DSL-Ansatz f�r gleich GUIs auf
unterschiedlichen Plattformen}{neuerAnsatz}

\section{Vorteile gegen�ber dem Multichannel-Framework}
Wie in Kapitel \ref{Probleme} erl�utert, wei�t das MCF einige Probleme auf. Mit
dem neuen Ansatz kann das Problem der inaktuellen Frameworks und das Problem der
starken Orientiertung an Swing (oder an ein anderes Framework) beseitigt werden.
Eine DSL sollte sich nicht an Besonderheiten bestehender
Frameworks orientieren, sondern an dem Dom�nenproblem. \cite[S.15]{mdsdvoelter}
Von daher sollte bei korrekter Umsetzung sichergestellt sein, dass die
Integration von unterschiedlichen Frameworks gleicher Ma�en gut funktioniert.\\
Ein weiterer Vorteil ist, dass durch die wegfallende Orientiertung an Swing auch
die Beschreibungsform ausdrucksst�rker wird. Grund daf�r ist, dass die
syntaktischen Strukten, die in Swing vorhanden sind, nicht mehr ben�tigt
werden.\\
Dazu kommt, dass die DSL so erweitert werden kann, dass fachliche Konzepte zur
Beschreibung der UIs benutzt werden k�nnen. Die Umsetzung dieser Konzepte
auf technischer Ebene w�rde von den Generatoren �bernommen werden und den
Entwickler nicht tangieren.\\
Weiterhin kann dem Entwickler durch die Codegenerierung die o.g. fehleranf�llige
Routinearbeit abgenommen werden.
