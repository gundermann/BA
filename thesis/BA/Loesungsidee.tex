\chapter{Entwicklung einer L�sungsidee}\label{Entwicklung einer L�sungsidee}
\section{Allgemeine Beschreibung der L�sungsidee}\label{AllgBeschreibL�sung}
Eine L�sungsidee f�r die in Kapitel \ref{Probleme} beschriebenen Probleme wurde
im Kapitel \ref{Ziel} bereits angedeutet. Kern dieser Idee ist, eine
DSL zur Beschreibung von \glspl{GUI} zu nutzen. Diese \glspl{GUI} sollen so
beschrieben werden, dass sie in der Dom�ne von profil c/s f�r unterschiedliche
\acrshort{GUI}-Frameworks genutzt werden k�nnen. Diese Beschreibung soll weiterhin nur
einmal stattfinden. Der Quell-Code, welcher die \acrshort{GUI} im entsprechenden
Framework darstellt, wird frameworkspezifisch aus der \acrshort{GUI}-Beschreibung generiert.
Langfristig betrachtet k�nnte das MCF damit abgel�st werden.\\
Der Prototyp, welcher in dieser Arbeit entwickelt wird, soll Quell-Code
erzeugen, der eine GUI mit den syntaktischen Strukturen des MCF beschreibt.
Somit kann gepr�ft werden, ob sich der generierte Quell-Code in die
Architektur der deg einbinden l�sst. 
\section{Architektur}
Die \g wird f�r die abstrakte
Beschreibung der \acrshort{GUI} verwendet. Somit ist gew�hrleistet, dass die \acrshort{GUI} weiterhin
nur einmal beschrieben werden muss. 
Wie in Abschnitt \ref{AllgBeschreibL�sung} beschrieben, wird der Quell-Code der \acrshort{GUI}
frameworkspezifisch, mit Hilfe eines speziellen Generators, erzeugt. Dadruch
ist die Integration neuer Frameworks an die Implementierung eines spezifischen
Generators gekoppelt. Abbildung \ref{Abb_neuerAnsatz} zeigt das grundlegende
Konzept f�r diesen Ansatz auf. Dabei wurden exemplarisch drei
unterschiedliche Generatoren f�r unterschiedliche Frameworks mit aufgenommen.
\myBigFigure{neuerAnsatz.png}{Grundlegendes Konzept}{neuerAnsatz}\noindent
Der Prototyp wird dabei aus einem \g-Skript Quell-Code erzeugen, welcher in das
MCF eingebunden werden kann. Somit l�sst sich pr�fen, ob die GUI-DSL f�r
existierende \glspl{GUI} von profil c/s gen�gt. In Anlehnung an Abbildung
\ref{Abb_mdsd} in Kapitel \ref{mdsd} wird eine \acrshort{GUI} mit sinnvollen
Interaktionen nicht allein �ber die \g nicht umsetzbar sein, da entsprechende
Modelle, �ber die Informationen durch die \acrshort{GUI} dem Nutzer kommuniziert
werden sollen, nicht in der \g beschrieben werden. Daher ist zumindest zur
Erzeugung diesre Informationesquellen individueller Quell-Code von N�ten.
Abbildung \ref{Abb_genMCFCode} zeigt diese grundlegende Idee schematisch auf.
\myBigFigure{genMCFCode.jpg}{Grundlegende Idee f�r den Prototypen}{genMCFCode}

